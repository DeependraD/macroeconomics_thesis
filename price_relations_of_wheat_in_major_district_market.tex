\PassOptionsToPackage{unicode=true}{hyperref} % options for packages loaded elsewhere
\PassOptionsToPackage{hyphens}{url}
\PassOptionsToPackage{dvipsnames,svgnames*,x11names*}{xcolor}
%
\documentclass[
  12pt,
]{article}
\usepackage{lmodern}
\usepackage{setspace}
\setstretch{1}
\usepackage{amssymb,amsmath}
\usepackage{ifxetex,ifluatex}
\ifnum 0\ifxetex 1\fi\ifluatex 1\fi=0 % if pdftex
  \usepackage[T1]{fontenc}
  \usepackage[utf8]{inputenc}
  \usepackage{textcomp} % provides euro and other symbols
\else % if luatex or xelatex
  \usepackage{unicode-math}
  \defaultfontfeatures{Scale=MatchLowercase}
  \defaultfontfeatures[\rmfamily]{Ligatures=TeX,Scale=1}
\fi
% use upquote if available, for straight quotes in verbatim environments
\IfFileExists{upquote.sty}{\usepackage{upquote}}{}
\IfFileExists{microtype.sty}{% use microtype if available
  \usepackage[]{microtype}
  \UseMicrotypeSet[protrusion]{basicmath} % disable protrusion for tt fonts
}{}
\makeatletter
\@ifundefined{KOMAClassName}{% if non-KOMA class
  \IfFileExists{parskip.sty}{%
    \usepackage{parskip}
  }{% else
    \setlength{\parindent}{0pt}
    \setlength{\parskip}{6pt plus 2pt minus 1pt}}
}{% if KOMA class
  \KOMAoptions{parskip=half}}
\makeatother
\usepackage{xcolor}
\IfFileExists{xurl.sty}{\usepackage{xurl}}{} % add URL line breaks if available
\IfFileExists{bookmark.sty}{\usepackage{bookmark}}{\usepackage{hyperref}}
\hypersetup{
  pdftitle={Price relations wheat in major domestic and international markets},
  pdfauthor={Samita Paudel},
  colorlinks=true,
  linkcolor=Maroon,
  filecolor=Maroon,
  citecolor=DodgerBlue4,
  urlcolor=Blue,
  breaklinks=true}
\urlstyle{same}  % don't use monospace font for urls
\usepackage[margin=1in]{geometry}
\usepackage{longtable,booktabs}
% Allow footnotes in longtable head/foot
\IfFileExists{footnotehyper.sty}{\usepackage{footnotehyper}}{\usepackage{footnote}}
\makesavenoteenv{longtable}
\usepackage{graphicx,grffile}
\makeatletter
\def\maxwidth{\ifdim\Gin@nat@width>\linewidth\linewidth\else\Gin@nat@width\fi}
\def\maxheight{\ifdim\Gin@nat@height>\textheight\textheight\else\Gin@nat@height\fi}
\makeatother
% Scale images if necessary, so that they will not overflow the page
% margins by default, and it is still possible to overwrite the defaults
% using explicit options in \includegraphics[width, height, ...]{}
\setkeys{Gin}{width=\maxwidth,height=\maxheight,keepaspectratio}
\setlength{\emergencystretch}{3em}  % prevent overfull lines
\providecommand{\tightlist}{%
  \setlength{\itemsep}{0pt}\setlength{\parskip}{0pt}}
\setcounter{secnumdepth}{5}
% Redefines (sub)paragraphs to behave more like sections
\ifx\paragraph\undefined\else
  \let\oldparagraph\paragraph
  \renewcommand{\paragraph}[1]{\oldparagraph{#1}\mbox{}}
\fi
\ifx\subparagraph\undefined\else
  \let\oldsubparagraph\subparagraph
  \renewcommand{\subparagraph}[1]{\oldsubparagraph{#1}\mbox{}}
\fi

% set default figure placement to htbp
\makeatletter
\def\fps@figure{htbp}
\makeatother

\usepackage{fancyhdr}
\pagestyle{fancy}
\fancyhead[L]{Samita Paudel}
\fancyhead[R]{Price relations of food commodities in major district ...}
\usepackage{lineno}
\usepackage{booktabs}
\usepackage{longtable}
\usepackage{array}
\usepackage{multirow}
\usepackage{wrapfig}
\usepackage{float}
\usepackage{colortbl}
\usepackage{pdflscape}
\usepackage{tabu}
\usepackage{threeparttable}
\usepackage{threeparttablex}
\usepackage[normalem]{ulem}
\usepackage{makecell}
\usepackage{xcolor}
\usepackage{coffee4}
\usepackage{tikz}
\usepackage{verbatim}
\usetikzlibrary{arrows,shapes}
\newcommand{\coffa}{\cofeAm{1}{1.0}{0}{5.5cm}{3cm}}
\newcommand{\coffb}{\cofeBm{1}{1.2}{83}{2cm}{1cm}}

\title{Price relations wheat in major domestic and international markets}
\author{Samita Paudel}
\date{6/28/2019}

\begin{document}
\maketitle

\hypertarget{notes}{%
\section{Notes}\label{notes}}

\begin{itemize}
\tightlist
\item
  Formulate simple linear regression
\item
  Multicollinearlity check\ldots{}refer to Variance Inflation Factor
\item
  Plot time series
\item
  Autocorrelation check
\item
  Drop variables with high multicollinarity
\item
  Use

  \begin{itemize}
  \tightlist
  \item
    Darbin-Watson test for autocorrelation test (D-statistic), or
  \item
    Brueuch Godfrey test for autocorrelation
  \end{itemize}
\item
  Use

  \begin{itemize}
  \tightlist
  \item
    Bursch-pagan test for heteroscadesticity
  \item
    Or plot histogram
  \end{itemize}
\item
  In time series analysis,

  \begin{itemize}
  \tightlist
  \item
    Use regression with OLS
  \item
    Plot data of each series (Fuel, prices, temperature)
  \item
    Test dickey fueller test (Use no trend, no drift), use all possibility
  \item
    Use augmented dickey fueller test if dickey fueller test does not capture the essense
  \item
    Perform detrending with first order difference
  \end{itemize}
\end{itemize}

\hypertarget{why-domestic-price-is-studied-at-district-level}{%
\section{Why domestic price is studied at district level}\label{why-domestic-price-is-studied-at-district-level}}

\begin{itemize}
\tightlist
\item
  Until recently, before federal structure of governance was into force, planning, budgeting, service delivery and policy interventions in Agriculture were all excised through district level bodies. Each district was itself accountable for market information accrual and reporting. Hence, most reliable form of price series data would be district level itself.
\item
  Districts present isolated markets and well organized customer segments surrounding that. For e.g., food grain retail market of kathmandu is drastically different from that of kailali, because while in the former consumer segment has a larger role to play in determining of market demand, the market of farwestern terai region has significant share of producer segments in determining what and when to produce.
\item
  District present different socio-economic narrative for food commodity trade which is heavily affected by the geographical context of the district itself. For e.g., Rupandehi market is more closely tied to bordering Indian market, because of minimal to none customs intervention in cross-country trade of food grain. The price effects of indian districts are more easily reflected in border region market prices, than in distant market such as Jumla and Surkhet.
\item
  District markets information systems are more organized than local level markets, mostly because there are factors that buffer price volatility in district level. For e.g., government intervene through subsidized input supply, facilitated by district level agriculture offices when prices of agricultural inputs are hightened. Also service delivery system, for example that providing subsidized farmers loan, is carried out at district level.
\end{itemize}

\hypertarget{why-only-few-domestic-markets-were-selected-for-study}{%
\section{Why only few domestic markets were selected for study}\label{why-only-few-domestic-markets-were-selected-for-study}}

\begin{itemize}
\tightlist
\item
  These markets either have a large production volumes or strong consumer segment. Because districts of terai, and mostly those of Central to Farwestern region, show consistently high annual production volume (Terai is also dubbed grain basket of Nepal) major producer districts in terai -- Kailali, Rupandehi, Parsa are included in the study. At the same time Chitwan and Kathmandu districts have prominance of consumers.
\item
  Some of the features that justify suitability of inclusion of abovementioned districts are presented:

  \begin{itemize}
  \tightlist
  \item
    Kailali district lies in farwestern region. It borders with India through Uttar Pradesh state. The district has huge chunks of land annually allocated to Wheat production (How much in total ???, what percentage of total wheat cultivated area ???). The farmer segment comprises mainly of Tharu community.
  \item
    Rupandehi district is located in Western terai region. The district adjoining with Indian market of Uttar Pradesh state, likewise, has large volume of grain production arising from Wheat cultivation.
  \item
    Parsa district lies in Central terai part of Nepal. It border with India through Bihar state and the district cultivates Wheat in large volume (areawise how much ???).
  \end{itemize}
\item
  Kathmandu and Chitwan markets are mostly dominanted by consumer segment, hence the retail price series of these districts are expected to differ from that of producer market districts.
\item
  ???Some of the food market features of Kathmandu and Surkhet districts???
\item
  Treatment of isolated markets into arbitrary category of consumer and producer districts will provide a more complete picture of the situation of region they represent -- Farwestern terai, Western-central terai and Eastern-central terai regions, respectively. This form of classification is expected to improve interpretability and overall increase predictive accuray of model in the face of price shocks.
\end{itemize}

\hypertarget{dependent-variables}{%
\section{Dependent variables}\label{dependent-variables}}

\hypertarget{retail-price-of-wheat-in-major-domestic-markets}{%
\subsection{Retail price of wheat in major domestic markets}\label{retail-price-of-wheat-in-major-domestic-markets}}

Price series of domestic markets were selected for study. The data represent imbalanced series of following 5 districts:

Kailali, Rupandehi, Parsa, Kathmandu, Chitwan

\includegraphics{price_relations_of_wheat_in_major_district_market_files/figure-latex/foreign-exchange-data-1.pdf}

\includegraphics{price_relations_of_wheat_in_major_district_market_files/figure-latex/cost-of-production-1.pdf}

\hypertarget{combined-series}{%
\subsection{Combined series}\label{combined-series}}

\includegraphics{price_relations_of_wheat_in_major_district_market_files/figure-latex/price-series-all-1.pdf} \includegraphics{price_relations_of_wheat_in_major_district_market_files/figure-latex/price-series-all-2.pdf}

\hypertarget{geographical-context-of-study-districts}{%
\section{Geographical context of study districts}\label{geographical-context-of-study-districts}}

\hypertarget{study-districts-and-market-centres}{%
\subsection{Study districts and market centres}\label{study-districts-and-market-centres}}

A map of study districts.

\begin{figure}
\includegraphics[width=0.95\linewidth]{price_relations_of_wheat_in_major_district_market_files/figure-latex/study-districts-1} \caption{Geographical context of selected district markets}\label{fig:study-districts}
\end{figure}

\hypertarget{aggregate-series-summary}{%
\subsection{Aggregate series summary}\label{aggregate-series-summary}}

Joint time series plot of rice and wheat retail prices aggregated over selected districtwise markets and comparison to national average price.

\includegraphics{price_relations_of_wheat_in_major_district_market_files/figure-latex/price-series-line-acf-plots-1.pdf} \includegraphics{price_relations_of_wheat_in_major_district_market_files/figure-latex/price-series-line-acf-plots-2.pdf} \includegraphics{price_relations_of_wheat_in_major_district_market_files/figure-latex/price-series-line-acf-plots-3.pdf} \includegraphics{price_relations_of_wheat_in_major_district_market_files/figure-latex/price-series-line-acf-plots-4.pdf} \includegraphics{price_relations_of_wheat_in_major_district_market_files/figure-latex/price-series-line-acf-plots-5.pdf}

Time series plot of wheat retail price series is presented in above figures with some diagnostic plot alongside. The lineplot shows that there exist some time gaps at random periods. Autocorrelation of consecutive first order differenced lags is similarly shown.

\hypertarget{linear-regression-model-formulation-for-price-series}{%
\section{Linear regression model formulation for price series}\label{linear-regression-model-formulation-for-price-series}}

\begin{table}[H]

\caption{\label{tab:lm1-without-time}ANOVA of regression between price of producer districts (as dependent variable) and 6 regressor variables (price of wheat in international market, price of wheat in indian bordering states, price of wheat in consumer districts, precipitation of consumer districts, precipitation of producer districts)}
\centering
\begin{tabular}[t]{>{\raggedright\arraybackslash}p{7em}rrrrr}
\toprule
term & df & sumsq & meansq & statistic & p.value\\
\midrule
price canada & 1 & 736.012 & 736.012 & 196.169 & 0.000\\
price india & 1 & 7604.395 & 7604.395 & 2026.798 & 0.000\\
price nepal chitwan and kathmandu & 1 & 607.508 & 607.508 & 161.919 & 0.000\\
precipitation total nepal chitwan and kathmandu & 1 & 0.070 & 0.070 & 0.019 & 0.891\\
precipitation total nepal kailali rupandehi parsa & 1 & 0.001 & 0.001 & 0.000 & 0.990\\
\addlinespace
fuel price & 1 & 139.376 & 139.376 & 37.148 & 0.000\\
Residuals & 150 & 562.789 & 3.752 & NA & NA\\
\bottomrule
\end{tabular}
\end{table}

The regression above (Table \ref{tab:lm1-without-time}) is obtained on fitting the model Equation \ref{eqn:lm1}.

\begin{equation}
\label{eqn:lm1}
\begin{aligned}
\begin{split}
price_{\textrm{producer districts}} &= price_{\textrm{canada}} + price_{\textrm{indian bordering states}} \\ &+
price_{\textrm{consumer districts}} + precipitation_{\textrm{producer districts}} \\ &+
precipitation_{\textrm{consumer districts}} + fuel~price_{\textrm{national average}}
\end{split}
\end{aligned}
\end{equation}

This presents a typical case of spurious relationship among variables, where variables having times series attributes show exceptionally high association among them. For example, all three aggregated wheat price series we consider in the regression (price in Canada, price in India, and price in Nepalese consumer markets) which show highly significant association. This is problematic and misleading, because without accounting for linear time trend they show very unstable variance, which increases at extremes (more recent time period). The phenomena of possible presence of heteroskedasticity is shown in the residual-vs-fit plot in Figure \ref{fig:lm1-residual-vs-fit-plot}.

\begin{figure}

{\centering \includegraphics{price_relations_of_wheat_in_major_district_market_files/figure-latex/lm1-residual-vs-fit-plot-1} 

}

\caption{Residual (pearsons') versus fit plot of the linear regression without accounting for time attributes of the series}\label{fig:lm1-residual-vs-fit-plot}
\end{figure}

Simply after incorporating linear trend of time in the regression model (Equation \ref{eqn:lm1}), we get a different statistic.

\begin{equation}
\label{eqn:lm2}
\begin{aligned}
\begin{split}
price_{\textrm{producer districts}} &= date + price_{\textrm{canada}} + price_{\textrm{indian bordering states}} \\ &+
price_{\textrm{consumer districts}} + precipitation_{\textrm{producer districts}} \\ &+
precipitation_{\textrm{consumer districts}} + fuel~price_{\textrm{national average}}
\end{split}
\end{aligned}
\end{equation}

However, even then there is likely presence of biased variance, as shown in Figure \ref{fig:lm2-residual-vs-fit-plot} which uses Equation \ref{eqn:lm2} for model fitting.

\begin{figure}

{\centering \includegraphics{price_relations_of_wheat_in_major_district_market_files/figure-latex/lm2-residual-vs-fit-plot-1} 

}

\caption{Residual (pearsons') versus fit plot of the linear regression after incorporating linear time trend of the price series}\label{fig:lm2-residual-vs-fit-plot}
\end{figure}

\hypertarget{testing-for-heteroskedasticity}{%
\subsection{Testing for heteroskedasticity}\label{testing-for-heteroskedasticity}}

Below we test the linear model (Equation \ref{eqn:lm2} and its counterpart with consumer districts as dependent variable and same independent variables with linear time trend) for presence of heteroskedasticity using Breusch-Pagan test. The test fits a linear regression model to the residuals of a linear regression model (by default the same explanatory variables are taken as in the main regression model) and rejects if too much of the variance is explained by the additional explanatory variables.

Breusch pagan test uses the null hypothesis of Homoscadesticity while testing studentized residuals. Hence, if the null hypothesis is rejected (p \textless{} 0.05), there is possible presence of heteroskedasticity.

\begin{table}

\caption{\label{tab:lm2-bptest}Breusch-Pagan test for heteroskedasticity of price series, modeled by the regression Equation ef(eqn:lm2) for two domestic price series (producers districts and consumer districts)}
\centering
\begin{tabular}[t]{rrrl}
\toprule
statistic & p.value & parameter & method\\
\midrule
19.65260 & 0.0063710 & 7 & studentized Breusch-Pagan test\\
26.73769 & 0.0003715 & 7 & studentized Breusch-Pagan test\\
\bottomrule
\end{tabular}
\end{table}

A possible measure to removing non-stationary trend in the series is by differencing (with \texttt{diff}). However, before progressing we confirm that justifiable lag operations can infact render the series free of trends. For this, two popular unit test routines are performed -- Augmented Dickey-Fueller test and KPSS test.

\hypertarget{unit-root-testing}{%
\section{Unit root testing}\label{unit-root-testing}}

\hypertarget{unit-root-adf-and-kpss-test-of-retail-price}{%
\subsection{Unit root (ADF and KPSS) test of retail price}\label{unit-root-adf-and-kpss-test-of-retail-price}}

The ADF, available in the function \texttt{adf.test()} (in the package \texttt{tseries}) implements the t-test of \(H_0: \gamma = 0\) in the regression, below.

\begin{equation}
\label{eqn:lagged-ts-regression}
  \Delta {{Y}_{t}}={{\beta
  }_{1}}+{{\beta }_{2}}t+\gamma {{Y}_{t-1}}+ \sum\limits_{i=1}^{m}{\delta_i \Delta
    {{Y}_{t-i}}+{{\varepsilon }_{t}}}
\end{equation}

The null is therefore that x has a unit root. If only x has a non-unit root, then the x is stationary (rejection of null hypothesis).

\begin{table}

\caption{\label{tab:adf-kpss-test-retail}Unit root test of log(price) series variables (both dependent and independent)}
\centering
\begin{tabular}[t]{>{\raggedright\arraybackslash}p{8em}lrrl}
\toprule
series & test & lprice pvalue & lprice tstatistic & lprice null accepted\\
\midrule
price canada & adf & 0.42 & -2.37 & TRUE\\
price india & adf & 0.16 & -3.00 & TRUE\\
price nepal chitwan and kathmandu & adf & 0.44 & -2.32 & TRUE\\
price nepal kailali rupandehi parsa & adf & 0.26 & -2.75 & TRUE\\
fuel price & adf & 0.09 & -3.16 & TRUE\\
\addlinespace
price canada & kpss & 0.01 & 1.60 & FALSE\\
price india & kpss & 0.01 & 4.29 & FALSE\\
price nepal chitwan and kathmandu & kpss & 0.01 & 3.47 & FALSE\\
price nepal kailali rupandehi parsa & kpss & 0.01 & 4.03 & FALSE\\
fuel price & kpss & 0.01 & 2.01 & FALSE\\
\bottomrule
\end{tabular}
\end{table}

The ADF test was parametrized with the alternative hypothesis of stationarity. This extends to following assumption in the model parameters;

\[
-2 \leq \gamma \leq 0\ \text{or } (-1 < 1+\phi < 1)
\]

\texttt{k} in the function refers to the number of \(\delta\) lags, i.e., \(1, 2, 3, ...., m\) in the model equation.

The number of lags \texttt{k} defaults to \texttt{trunc((length(x)-1)\^{}(1/3))}, where \texttt{x} is the series being tested. The default value of \texttt{k} corresponds to the suggested upper bound on the rate at which the number of lags, \texttt{k}, should be made to grow with the sample size for the general ARMA(p,q) setup \texttt{citation(package\ =\ "tseries")}.

For a Dickey-Fueller test, so only up to AR(1) time dependency in our stationary process, we set \texttt{k\ =\ 0}. Hence we have no \(\delta\)s (lags) in our test.

The DF model can be written as:

\[
Y_t = \beta_1 + \beta_2 t + \phi Y_{t-1} + \varepsilon_t
\]

It can be re-written so we can do a linear regression of \(\Delta Y_t\) against \(t\) and \(Y_{t-1}\) and test if \(\phi\) is different from 0. If only, \(\phi\) is not zero and assumption above (\(-1 < 1+\phi < 1\)) holds, the process is stationary. If \(\phi\) is straight up 0, then we have a random walk process -- all white noise.

\[
\Delta {Y}_{t}=\beta_1+\beta_2 t+\gamma {Y}_{t-1} + \varepsilon_{t}
\]

Alternative to above discussed tests, the Phillips-Perron test with its nonparametric correction for autocorrelation (essentially employing a HAC estimate of the long-run variance in a Dickey-Fuller-type test instead of parametric decorrelation) may be used. It is available in the function \texttt{pp.test()}.

\hypertarget{unit-root-test-based-lag-order-differencing-determination}{%
\subsection{Unit root test based lag order differencing determination}\label{unit-root-test-based-lag-order-differencing-determination}}

An alternative to decomposition for removing trends is differencing (Woodward, Gray, and Elliott \protect\hyperlink{ref-woodward2017applied}{2017}). We define the difference operator as,

\begin{equation}
\nabla x_t = x_t - x_{t-1},
\label{eqn:difference-operator}
\end{equation}

and, more generally, for order \(d\)

\begin{equation}
\nabla^d x_t = (1-\mathbf{B})^d x_t,
\label{eqn:order-d-difference-operator}
\end{equation}

Where \(\mathbf{B}\) is the backshift operator (i.e., \(\mathbf{B}^k x_t = x_{t-k}\) for \(k \geq 1\)).

Applying the difference to a random walk, the most simple and widely used time series model, will yield a time series of Gaussian white noise errors \(\{w_t\}\):

\begin{equation}
  \begin{aligned}
    \nabla (x_t &= x_{t-1} + w_t) \\
    x_t - x_{t-1} &= x_{t-1} - x_{t-1} + w_t \\
    x_t - x_{t-1} &= w_t
  \end{aligned}
  \label{eqn:random-walk-series}
\end{equation}

We use an implementation of time series differencing based on optimal lag length in order to render series stationary. The first lag order differenced log(price) series (derived based on ``kpss'' test statistic) is shown in Figure \ref{fig:differenced-series-kpss-lag}. The kpss test, however, determined the precipitation series to be integrated of order null. This is due to insensitivity of test (model) to seasonal lag component, which is infact nicely captured by addtive or multiplicative trend decomposition (discussed ahead).

\begin{figure}
\centering
\includegraphics{price_relations_of_wheat_in_major_district_market_files/figure-latex/differenced-series-kpss-lag-1.pdf}
\caption{\label{fig:differenced-series-kpss-lag}Plot of differenced time series for various lag length determined by kpss statistic.}
\end{figure}

The first order differencing of price renders series stationary. However, precipitation series shows distinct components. Here, a linear model is fitted to the trend component, decomposed (into seasonal and trend components) from two precipitation series (producer and consumer districts), along with other terms of Equation \ref{eqn:lm2} and the ANOVA output is presented.

\begin{table}

\caption{\label{tab:anova-producer-market}ANOVA of linear model fit with trend component of precipitation series with price(producer market) as dependent variable.}
\centering
\begin{tabular}[t]{lrrrrr}
\toprule
term & df & sumsq & meansq & statistic & p.value\\
\midrule
date & 1 & 8810.50 & 8810.50 & 2463.40 & 0.00\\
price\_canada & 1 & 16.17 & 16.17 & 4.52 & 0.04\\
price\_india & 1 & 12.83 & 12.83 & 3.59 & 0.06\\
price\_nepal\_chitwan\_and\_kathmandu & 1 & 145.70 & 145.70 & 40.74 & 0.00\\
precipitation\_total\_ck\_trend & 1 & 10.54 & 10.54 & 2.95 & 0.09\\
\addlinespace
precipitation\_total\_krp\_trend & 1 & 33.17 & 33.17 & 9.27 & 0.00\\
fuel\_price & 1 & 88.33 & 88.33 & 24.70 & 0.00\\
Residuals & 149 & 532.91 & 3.58 & NA & NA\\
\bottomrule
\end{tabular}
\end{table}

\begin{table}

\caption{\label{tab:anova-consumer-market}ANOVA of linear model fit with trend component of precipitation series with price(consumer market) as dependent variable.}
\centering
\begin{tabular}[t]{lrrrrr}
\toprule
term & df & sumsq & meansq & statistic & p.value\\
\midrule
date & 1 & 14695.17 & 14695.17 & 2146.39 & 0.00\\
price\_canada & 1 & 0.41 & 0.41 & 0.06 & 0.81\\
price\_india & 1 & 50.51 & 50.51 & 7.38 & 0.01\\
price\_nepal\_kailali\_rupandehi\_parsa & 1 & 243.06 & 243.06 & 35.50 & 0.00\\
precipitation\_total\_ck\_trend & 1 & 0.45 & 0.45 & 0.07 & 0.80\\
\addlinespace
precipitation\_total\_krp\_trend & 1 & 72.78 & 72.78 & 10.63 & 0.00\\
fuel\_price & 1 & 15.89 & 15.89 & 2.32 & 0.13\\
Residuals & 149 & 1020.12 & 6.85 & NA & NA\\
\bottomrule
\end{tabular}
\end{table}

\hypertarget{var-model}{%
\section{VAR model}\label{var-model}}

\hypertarget{var}{%
\subsection{VAR}\label{var}}

VAR is a system regression model, i.e., there are more than one dependent variable. The regression is defined by a set of linear dynamic equations where each variable is specified as a function of an equal number of lags of itself and all other variables in the system. Any additional variable, adds to the modeling complexity by increasing an extra equation to be estimated.

The vector autoregression (VAR) model extends the idea of univariate autoregression to \(k\) time series regressions, where the lagged values of \emph{all} \(k\) series appear as regressors. Put differently, in a VAR model we regress a \emph{vector} of time series variables on lagged vectors of these variables. As for \(AR(p)\) models, the lag order is denoted by \(p\) so the \(VAR(p)\) model of two variables \(X_t\) and \(Y_t\) (\(k=2\)) is given by a vector of equations (Equation \ref{eqn:vector-regression-ts}).

\begin{equation}
\label{eqn:vector-regression-ts}
\begin{split}
\begin{aligned}
  Y_t =& \, \beta_{10} + \beta_{11} Y_{t-1} + \dots + \beta_{1p} Y_{t-p} + \gamma_{11} X_{t-1} + \dots + \gamma_{1p} X_{t-p} + u_{1t}, \\
  X_t =& \, \beta_{20} + \beta_{21} Y_{t-1} + \dots + \beta_{2p} Y_{t-p} + \gamma_{21} X_{t-1} + \dots + \gamma_{2p} X_{t-p} + u_{2t}.
\end{aligned}
\end{split}
\end{equation}

The \(\beta\)s and \(\gamma\)s can be estimated using OLS on each equation.

Simplifying this to a bivariate \(VAR(1)\), we can write the model in matrix form as:

\begin{equation}
\label{eqn:matix-var1-model}
Y_t = \beta_0 + \beta_1 Y_{t-1} + \mu_t
\end{equation}

Where,

\begin{itemize}
\tightlist
\item
  \(Y_t, Y_{t-1}\) and \(\mu_t\) are (2 x 1) column vectors
\item
  \(\beta_0\) is a (2 x 1) column vector
\item
  \(\beta_1\) is a (2 x 2) matrix
\end{itemize}

also,

\[
Y_t = 
\begin{pmatrix} 
y_{1t} \\
y_{2t}
\end{pmatrix},\ 
Y_{t-1} = 
\begin{pmatrix} 
y_{1t-1} \\
y_{2t-1}
\end{pmatrix}
\]

\[
\mu_t = 
\begin{pmatrix} 
\mu_{1t} \\
\mu_{2t}
\end{pmatrix},
\beta_{0} = 
\begin{pmatrix} 
\beta_{10} \\
\beta_{20}
\end{pmatrix},
\beta_{1} = 
\begin{pmatrix} 
\beta_{11} & \alpha_{11} \\
\alpha_{21} & \beta_{21}
\end{pmatrix}
\]

It is straightforward to estimate VAR models in \texttt{R}. A feasible approach is to simply use \texttt{lm()} for estimation of the individual equations. Furthermore, the \texttt{vars} package provides standard tools for estimation, diagnostic testing and prediction using this type of models.

Only when the assumptions presented below hold, the OLS estimators of the VAR coefficients are consistent and jointly normal in large samples so that the usual inferential methods such as confidence intervals and \(t\)-statistics can be used (Metcalfe and Cowpertwait \protect\hyperlink{ref-metcalfe2009introductory}{2009}).

Two series \(w_{x,t}\) and \(w_{y,t}\) are bivariate white noise if they are stationary and their cross-covariances \(\gamma_{xy}(k) = Cov(w_{x,t}, w_{y, t+k})\) satisfies

\[
\gamma_{xx}(k) = \gamma_{yy}(k) = \gamma_{xy}(k) = 0\ \text{for all } k \neq 0
\]

The parameters of a var(p) model can be estimated using the \texttt{ar} function in \texttt{R}, which selects a best-fitting order \(p\) based on the smallest information criterion values.

The structure of VARs also allows to jointly test restrictions across multiple equations. For instance, it may be of interest to test whether the coefficients on all regressors of the lag \(p\) are zero. This corresponds to testing the null that the lag order \(p-1\) is correct. Large sample joint normality of the coefficient estimates is convenient because it implies that we may simply use an \(F\)-test for this testing problem. The explicit formula for such a test statistic is rather complicated but fortunately such computations are easily done using the \texttt{ttcode("R")} functions we work with in this chapter. Just as in the case of a single equation, for a multiple equation model we choose the specification which has the smallest \(BIC(p)\), where

\[
\begin{aligned}
  BIC(p) =& \, \log\left[\text{det}(\widehat{\Sigma}_u)\right] + k(kp+1) \frac{\log(T)}{T}.
\end{aligned}
\]

with \(\widehat{\Sigma}_u\) denoting the estimate of the \(k \times k\) covariance matrix of the VAR errors and \(\text{det}(\cdot)\) denotes the determinant.

As for univariate distributed lag models, one should think carefully about variables to include in a VAR, as adding unrelated variables reduces the forecast accuracy by increasing the estimation error. This is particularly important because the number of parameters to be estimated grows qudratically to the number of variables modeled by the VAR.

\begin{longtable}[t]{llrrrr}
\caption{\label{tab:retail-var-fit-tidy}Model coefficients of VAR(AR(1)) model for for wheat log(price) series with two domestic (producer and consumer) markets as endogeneous variables and other price series, precipitation and fuel series as exogeneous regressors.}\\
\toprule
term & .response & estimate & std.error & statistic & p.value\\
\midrule
lag(lp\_npl\_ck,1) & lp\_npl\_ck & 0.848 & 0.032 & 26.572 & 0.000\\
lag(lp\_npl\_krp,1) & lp\_npl\_ck & 0.055 & 0.041 & 1.356 & 0.176\\
constant & lp\_npl\_ck & 0.110 & 0.044 & 2.502 & 0.013\\
lp\_canada & lp\_npl\_ck & -0.003 & 0.015 & -0.231 & 0.818\\
lp\_india & lp\_npl\_ck & 0.096 & 0.037 & 2.578 & 0.011\\
\addlinespace
precip\_npl\_ck & lp\_npl\_ck & -0.002 & 0.004 & -0.520 & 0.604\\
precip\_npl\_krp & lp\_npl\_ck & 0.003 & 0.004 & 0.761 & 0.447\\
lp\_fuel & lp\_npl\_ck & -0.011 & 0.027 & -0.415 & 0.679\\
lag(lp\_npl\_ck,1) & lp\_npl\_krp & 0.033 & 0.030 & 1.100 & 0.273\\
lag(lp\_npl\_krp,1) & lp\_npl\_krp & 0.792 & 0.039 & 20.530 & 0.000\\
\addlinespace
constant & lp\_npl\_krp & -0.060 & 0.042 & -1.438 & 0.152\\
lp\_canada & lp\_npl\_krp & -0.003 & 0.014 & -0.205 & 0.838\\
lp\_india & lp\_npl\_krp & 0.102 & 0.035 & 2.914 & 0.004\\
precip\_npl\_ck & lp\_npl\_krp & -0.002 & 0.004 & -0.457 & 0.648\\
precip\_npl\_krp & lp\_npl\_krp & 0.002 & 0.003 & 0.700 & 0.485\\
\addlinespace
lp\_fuel & lp\_npl\_krp & 0.079 & 0.025 & 3.175 & 0.002\\
\bottomrule
\end{longtable}

--\textgreater{}

--\textgreater{}

--\textgreater{}
--\textgreater{}
--\textgreater{}
--\textgreater{}

--\textgreater{}

\hypertarget{bibliography}{%
\section*{Bibliography}\label{bibliography}}
\addcontentsline{toc}{section}{Bibliography}

\hypertarget{refs}{}
\leavevmode\hypertarget{ref-metcalfe2009introductory}{}%
Metcalfe, Andrew V, and Paul SP Cowpertwait. 2009. \emph{Introductory Time Series with R}. Springer.

\leavevmode\hypertarget{ref-woodward2017applied}{}%
Woodward, Wayne A, Henry L Gray, and Alan C Elliott. 2017. \emph{Applied Time Series Analysis with R}. CRC press.

\end{document}
