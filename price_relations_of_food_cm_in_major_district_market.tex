\PassOptionsToPackage{unicode=true}{hyperref} % options for packages loaded elsewhere
\PassOptionsToPackage{hyphens}{url}
\PassOptionsToPackage{dvipsnames,svgnames*,x11names*}{xcolor}
%
\documentclass[
  12pt,
]{article}
\usepackage{lmodern}
\usepackage{setspace}
\setstretch{1}
\usepackage{amssymb,amsmath}
\usepackage{ifxetex,ifluatex}
\ifnum 0\ifxetex 1\fi\ifluatex 1\fi=0 % if pdftex
  \usepackage[T1]{fontenc}
  \usepackage[utf8]{inputenc}
  \usepackage{textcomp} % provides euro and other symbols
\else % if luatex or xelatex
  \usepackage{unicode-math}
  \defaultfontfeatures{Scale=MatchLowercase}
  \defaultfontfeatures[\rmfamily]{Ligatures=TeX,Scale=1}
\fi
% use upquote if available, for straight quotes in verbatim environments
\IfFileExists{upquote.sty}{\usepackage{upquote}}{}
\IfFileExists{microtype.sty}{% use microtype if available
  \usepackage[]{microtype}
  \UseMicrotypeSet[protrusion]{basicmath} % disable protrusion for tt fonts
}{}
\makeatletter
\@ifundefined{KOMAClassName}{% if non-KOMA class
  \IfFileExists{parskip.sty}{%
    \usepackage{parskip}
  }{% else
    \setlength{\parindent}{0pt}
    \setlength{\parskip}{6pt plus 2pt minus 1pt}}
}{% if KOMA class
  \KOMAoptions{parskip=half}}
\makeatother
\usepackage{xcolor}
\IfFileExists{xurl.sty}{\usepackage{xurl}}{} % add URL line breaks if available
\IfFileExists{bookmark.sty}{\usepackage{bookmark}}{\usepackage{hyperref}}
\hypersetup{
  pdftitle={Price relations of food commodities in regional markets of Nepal},
  pdfauthor={Samita Paudel},
  colorlinks=true,
  linkcolor=Maroon,
  filecolor=Maroon,
  citecolor=DodgerBlue4,
  urlcolor=Blue,
  breaklinks=true}
\urlstyle{same}  % don't use monospace font for urls
\usepackage[margin=1in]{geometry}
\usepackage{longtable,booktabs}
% Allow footnotes in longtable head/foot
\IfFileExists{footnotehyper.sty}{\usepackage{footnotehyper}}{\usepackage{footnote}}
\makesavenoteenv{longtable}
\usepackage{graphicx,grffile}
\makeatletter
\def\maxwidth{\ifdim\Gin@nat@width>\linewidth\linewidth\else\Gin@nat@width\fi}
\def\maxheight{\ifdim\Gin@nat@height>\textheight\textheight\else\Gin@nat@height\fi}
\makeatother
% Scale images if necessary, so that they will not overflow the page
% margins by default, and it is still possible to overwrite the defaults
% using explicit options in \includegraphics[width, height, ...]{}
\setkeys{Gin}{width=\maxwidth,height=\maxheight,keepaspectratio}
\setlength{\emergencystretch}{3em}  % prevent overfull lines
\providecommand{\tightlist}{%
  \setlength{\itemsep}{0pt}\setlength{\parskip}{0pt}}
\setcounter{secnumdepth}{5}
% Redefines (sub)paragraphs to behave more like sections
\ifx\paragraph\undefined\else
  \let\oldparagraph\paragraph
  \renewcommand{\paragraph}[1]{\oldparagraph{#1}\mbox{}}
\fi
\ifx\subparagraph\undefined\else
  \let\oldsubparagraph\subparagraph
  \renewcommand{\subparagraph}[1]{\oldsubparagraph{#1}\mbox{}}
\fi

% set default figure placement to htbp
\makeatletter
\def\fps@figure{htbp}
\makeatother

\usepackage{fancyhdr}
\pagestyle{fancy}
\fancyhead[L]{Samita Paudel}
\fancyhead[R]{Price relations of food commodities in major district ...}
\usepackage{lineno}
\usepackage{booktabs}
\usepackage{longtable}
\usepackage{array}
\usepackage{multirow}
\usepackage{wrapfig}
\usepackage{float}
\usepackage{colortbl}
\usepackage{pdflscape}
\usepackage{tabu}
\usepackage{threeparttable}
\usepackage{threeparttablex}
\usepackage[normalem]{ulem}
\usepackage{makecell}
\usepackage{xcolor}
\usepackage{coffee4}
\usepackage{tikz}
\usepackage{verbatim}
\usetikzlibrary{arrows,shapes}
\newcommand{\coffa}{\cofeAm{1}{1.0}{0}{5.5cm}{3cm}}
\newcommand{\coffb}{\cofeBm{1}{1.2}{83}{2cm}{1cm}}

\title{Price relations of food commodities in regional markets of Nepal}
\author{Samita Paudel}
\date{6/28/2019}

\begin{document}
\maketitle

\hypertarget{retail-price-of-rice-and-wheat-in-major-districtwise-nepalese-market-hubs}{%
\section{Retail price of rice and wheat in major districtwise Nepalese market hubs}\label{retail-price-of-rice-and-wheat-in-major-districtwise-nepalese-market-hubs}}

The price series of districtwise market hubs available for study is mostly imbalanced and irregular and contains data for following 21 districts.

Achham, Banke, Bhojpur, Chitwan, Dhankuta, Dhanusha, Doti, Illam, Jhapa, Jumla, Kailali, Kaski, Kathmandu, Morang, Nuwakot, Palpa, Parsa, Ramechap, Rolpa, Rupandehi, Surkhet.

--\textgreater{}

\hypertarget{aggregate-series-summary}{%
\subsection{Aggregate series summary}\label{aggregate-series-summary}}

Joint time series plot of rice and wheat retail prices aggregated over selected districtwise markets.

\includegraphics{price_relations_of_food_cm_in_major_district_market_files/figure-latex/aggregated-price-summary-1.pdf}

Time series plot of retail price of rice in Kathmandu market. Series has some time gaps at random periods (shown on line plot on the lower right). Similarly, autocorrelation of series for various lag, with first order difference, is presented in the lower left.

\includegraphics{price_relations_of_food_cm_in_major_district_market_files/figure-latex/retail-kathmandu-series-exploratory-1.pdf}

A possible measure to removing non-stationary trend in the series is by differencing (with \texttt{diff}) . However, before progressing we confirm that justifiable lag operations can infact render the series free of trends. For this at two fundamentally different unit tests are performed -- Augmented Dickey-Fueller test and KPSS test.

While the series needs detrending in order to perform regression, we also have to consider the time gaps in the available dataset.

\includegraphics{price_relations_of_food_cm_in_major_district_market_files/figure-latex/time-gaps-1.pdf}

All district market series presented after time gap filling. First order lag is used to fill the missing entries in the series. Missing values for each series are imputed independent of other series.

\includegraphics{price_relations_of_food_cm_in_major_district_market_files/figure-latex/time-gap-filling-1.pdf}

\hypertarget{unit-root-testing}{%
\section{Unit root testing}\label{unit-root-testing}}

\hypertarget{unit-root-adf-and-kpss-test-of-retail-price}{%
\subsection{Unit root (ADF and KPSS) test of retail price}\label{unit-root-adf-and-kpss-test-of-retail-price}}

The ADF, available in the function \texttt{adf.test()} (in the package \texttt{tseries}) implements the t-test of \(H_0: \gamma = 0\) in the regression, below.

\begin{equation}
\label{eqn:lagged-ts-regression}
  \Delta {{Y}_{t}}={{\beta
  }_{1}}+{{\beta }_{2}}t+\gamma {{Y}_{t-1}}+ \sum\limits_{i=1}^{m}{\delta_i \Delta
    {{Y}_{t-i}}+{{\varepsilon }_{t}}}
\end{equation}

The null is therefore that x has a unit root. If only x has a non-unit root, then the x is stationary (rejection of null hypothesis).

\begin{table}

\caption{\label{tab:adf-kpss-test-retail}Unit root test of district market retail log(price) data of Rice}
\centering
\begin{tabular}[t]{>{\raggedright\arraybackslash}p{4em}>{\raggedright\arraybackslash}p{4em}>{\raggedright\arraybackslash}p{4em}>{\raggedright\arraybackslash}p{4em}>{\raggedright\arraybackslash}p{4em}>{\raggedright\arraybackslash}p{4em}>{\raggedright\arraybackslash}p{4em}}
\toprule
mktname & adf pvalue & adf tstatistic & adf stationary & kpss pvalue & kpss tstatistic & kpss stationary\\
\midrule
Banke & 0.22 & -2.86 & FALSE & 0.01 & 3.70 & FALSE\\
Kailali & 0.52 & -2.13 & FALSE & 0.01 & 3.72 & FALSE\\
Kaski & 0.86 & -1.33 & FALSE & 0.01 & 3.20 & FALSE\\
Kathmandu & 0.45 & -2.30 & FALSE & 0.01 & 3.00 & FALSE\\
Morang & 0.44 & -2.32 & FALSE & 0.01 & 3.42 & FALSE\\
\addlinespace
Parsa & 0.22 & -2.86 & FALSE & 0.01 & 2.64 & FALSE\\
Rupandehi & 0.01 & -4.51 & TRUE & 0.01 & 3.91 & FALSE\\
\bottomrule
\end{tabular}
\end{table}
\begin{table}

\caption{\label{tab:adf-kpss-test-retail}Unit root test of district market retail log(price) data of Wheat}
\centering
\begin{tabular}[t]{>{\raggedright\arraybackslash}p{4em}>{\raggedright\arraybackslash}p{4em}>{\raggedright\arraybackslash}p{4em}>{\raggedright\arraybackslash}p{4em}>{\raggedright\arraybackslash}p{4em}>{\raggedright\arraybackslash}p{4em}>{\raggedright\arraybackslash}p{4em}}
\toprule
mktname & adf pvalue & adf tstatistic & adf stationary & kpss pvalue & kpss tstatistic & kpss stationary\\
\midrule
Banke & 0.01 & -4.09 & TRUE & 0.01 & 3.86 & FALSE\\
Kailali & 0.47 & -2.26 & FALSE & 0.01 & 3.81 & FALSE\\
Kaski & 0.05 & -3.41 & FALSE & 0.01 & 3.40 & FALSE\\
Kathmandu & 0.34 & -2.56 & FALSE & 0.01 & 3.43 & FALSE\\
Morang & 0.03 & -3.62 & TRUE & 0.01 & 3.40 & FALSE\\
\addlinespace
Parsa & 0.04 & -3.55 & TRUE & 0.01 & 3.05 & FALSE\\
Rupandehi & 0.02 & -3.72 & TRUE & 0.01 & 3.16 & FALSE\\
\bottomrule
\end{tabular}
\end{table}

The ADF test was parametrized with the alternative hypothesis of stationarity. This extends to following assumption in the model parameters;

\[
-2 \leq \gamma \leq 0\ \text{or } (-1 < 1+\phi < 1)
\]

\texttt{k} in the function refers to the number of \(\delta\) lags, i.e., \(1, 2, 3, ...., m\) in the model equation.

The number of lags \texttt{k} defaults to \texttt{trunc((length(x)-1)\^{}(1/3))}, where \texttt{x} is the series being tested. The default value of \texttt{k} corresponds to the suggested upper bound on the rate at which the number of lags, \texttt{k}, should be made to grow with the sample size for the general ARMA(p,q) setup \texttt{citation(package\ =\ "tseries")}.

For a Dickey-Fueller test, so only up to AR(1) time dependency in our stationary process, we set \texttt{k\ =\ 0}. Hence we have no \(\delta\)s (lags) in our test.

The DF model can be written as:

\[
Y_t = \beta_1 + \beta_2 t + \phi Y_{t-1} + \varepsilon_t
\]

It can be re-written so we can do a linear regression of \(\Delta Y_t\) against \(t\) and \(Y_{t-1}\) and test if \(\phi\) is different from 0. If only, \(\phi\) is not zero and assumption above (\(-1 < 1+\phi < 1\)) holds, the process is stationary. If \(\phi\) is straight up 0, then we have a random walk process -- all white noise.

\[
\Delta {Y}_{t}=\beta_1+\beta_2 t+\gamma {Y}_{t-1} + \varepsilon_{t}
\]

Alternative to above discussed tests, the Phillips-Perron test with its nonparametric correction for autocorrelation (essentially employing a HAC estimate of the long-run variance in a Dickey-Fuller-type test instead of parametric decorrelation) may be used. It is available in the function \texttt{pp.test()}.

\hypertarget{unit-root-test-based-lag-order-differencing-determination}{%
\subsection{Unit root test based lag order differencing determination}\label{unit-root-test-based-lag-order-differencing-determination}}

An alternative to decomposition for removing trends is differencing (Woodward, Gray, and Elliott \protect\hyperlink{ref-woodward2017applied}{2017}). We define the difference operator as,

\begin{equation}
\nabla x_t = x_t - x_{t-1},
\label{eqn:difference-operator}
\end{equation}

and, more generally, for order \(d\)

\begin{equation}
\nabla^d x_t = (1-\mathbf{B})^d x_t,
\label{eqn:order-d-difference-operator}
\end{equation}

Where \(\mathbf{B}\) is the backshift operator (i.e., \(\mathbf{B}^k x_t = x_{t-k}\) for \(k \geq 1\)).

Applying the difference to a random walk, the most simple and widely used time series model, will yield a time series of Gaussian white noise errors \(\{w_t\}\):

\begin{equation}
  \begin{aligned}
    \nabla (x_t &= x_{t-1} + w_t) \\
    x_t - x_{t-1} &= x_{t-1} - x_{t-1} + w_t \\
    x_t - x_{t-1} &= w_t
  \end{aligned}
  \label{eqn:random-walk-series}
\end{equation}

Differencing is required for \emph{all} series to make them stationary, as inferred by the \texttt{ndiffs} function which employed popular unit root tests. A detailed presentation of the test routines is given below. We describe in detail the ADF test, while only tabluation of summary statistics of \texttt{kpss} test is made herein.

\hypertarget{unit-root-tests-of-first-order-differenced-series}{%
\subsection{Unit root tests of first order differenced series}\label{unit-root-tests-of-first-order-differenced-series}}

All major districtwise market series series are non-stationary, meaning that they have a trend associated with time.

We test the logged prices of the series after first order differencing. Here we perform a more conservative Dickey-Fueller, instead of Augmented DF, test.

\includegraphics{price_relations_of_food_cm_in_major_district_market_files/figure-latex/adf-test-diffretail-1.pdf}

The first order differencing renders all series stationary.

\hypertarget{arima-model}{%
\section{ARIMA model}\label{arima-model}}

\hypertarget{ar-model}{%
\subsection{AR model}\label{ar-model}}

A simple way to model dependence between consecutive observations is by specification of model such as one in Equation \ref{eqn:ar1-model}.

\begin{equation}
\label{eqn:ar1-model}
y_t = \mu_0 + \phi_1 y_{t-1} + \varepsilon_t
\end{equation}

Where, \(\varepsilon_t\) is the white noise having constant mean and variance and is independently distributed, i.e., has no autocorrelation.

Model representation of equation \ref{eqn:ar1-model} is called a first-order autoregressive process (\(AR(1)\)), as it has one lag of \(y_t\).

A \(p^{th}\) order autoregressive model -- \(AR(p)\) can be defined as in Equation \ref{eqn:ar-process}.

\begin{equation}
\label{eqn:ar-process}
y_t = \mu_0 + \phi_1 y_{t-1} + \phi_2 y_{t-2} + ... + \phi_p y_{t - p} + \varepsilon_t
\end{equation}

Autoregressive process of order \((p)\) can also be represented using lag operator as in Equation \ref{eqn:ar-process-lag-notation}.

\begin{equation}
\label{eqn:ar-process-lag-notation}
y_t = \mu_0 + \sum_{i = 1}^{p}{\phi_i L^i y_t} + \varepsilon_t
\end{equation}

Where,

\[
L y_t = y_{t-1},\ L^2 y_t = y_{t-2},\ L^iy_t = y_{t-i}
\]

For an \(AR(1)\) model (Equation \ref{eqn:ar1-model}), following states may exist:

\begin{itemize}
\tightlist
\item
  When \(\phi_1 = 0\), \(y_t\) is equivalent to white noise;
\item
  When \(\phi_1 = 1\) and \(\mu_0 = 0\), \(y_t\) is equivalent to random walk;
\item
  When \(\phi_1 = 1\) and \(c \neq 0\), \(y_t\) is equivalent to a random walk with drift;
\item
  When \(\phi_1 < 0\), \(y_t\) tends to oscillate between positive and negative value.
\end{itemize}

\hypertarget{ma-model}{%
\subsection{MA model}\label{ma-model}}

An MA(q) process is

\begin{equation}
\label{eqn:ma-process}
y_t = c + \varepsilon_t + \theta_1 \varepsilon_{t-1} + \theta_2 \varepsilon_{t-2} + ... + \theta_q \varepsilon_{t-q}
\end{equation}

Following assumptions may hold for a MA(q) model

\begin{itemize}
\tightlist
\item
  For an MA(1) model: \(-1 < \theta_1 < 1\)
\item
  For an MA(2) model: \(-1 < \theta_2 < 1\), \(\theta_2 + \theta_1 > -1\), \(\theta_1 - \theta_2 < 1\)
\end{itemize}

A process \(x_t\) is said to be ARIMA(p, d, q) if

\begin{equation}
\label{eqn:arima}
\begin{aligned}
  \nabla^d x_t = ( 1- B)^d x_t
\end{aligned}
\end{equation}

if ARMA(p, q). In general, we write the total model as:

\begin{equation}
\label{eqn:arima-total}
\phi (B)(1-B)^d x_t = \theta (B)w_t
\end{equation}

If \(E (\nabla^d x_t) = \mu\), we write the model as:

\begin{equation}
\label{eqn:arima-full}
\phi(B)(1-B)^d x_t = \delta + \theta (B) w_t
\end{equation}

Where \(\delta = \mu (1-\phi_1 - ... - \phi_p )\).

\begin{table}

\caption{\label{tab:multiple-arima-summary}ARIMA model summary for multiple log(price) series of major districtwise market hubs of Rice - Retail}
\centering
\begin{tabular}[t]{lrrrrr}
\toprule
mktname & sigma2 & log\_lik & AIC & AICc & BIC\\
\midrule
Kathmandu & 0.004 & 243.522 & -479.045 & -478.825 & -466.120\\
Parsa & 0.010 & 167.945 & -331.890 & -331.824 & -325.438\\
Morang & 0.004 & 257.864 & -509.728 & -509.598 & -500.003\\
Kailali & 0.004 & 290.087 & -568.173 & -567.781 & -547.784\\
Banke & 0.005 & 265.152 & -520.305 & -520.027 & -503.291\\
\addlinespace
Kaski & 0.002 & 304.876 & -603.752 & -603.622 & -594.027\\
Rupandehi & 0.007 & 234.554 & -463.107 & -462.997 & -452.899\\
\bottomrule
\end{tabular}
\end{table}
\begin{table}

\caption{\label{tab:multiple-arima-summary}ARIMA model summary for multiple log(price) series of major districtwise market hubs of Wheat - Retail}
\centering
\begin{tabular}[t]{lrrrrr}
\toprule
mktname & sigma2 & log\_lik & AIC & AICc & BIC\\
\midrule
Kathmandu & 0.005 & 239.962 & -473.923 & -473.793 & -464.198\\
Parsa & 0.006 & 212.092 & -418.184 & -418.052 & -408.507\\
Morang & 0.003 & 282.470 & -554.939 & -554.611 & -538.730\\
Kailali & 0.004 & 293.371 & -578.742 & -578.558 & -565.131\\
Banke & 0.004 & 291.350 & -568.700 & -568.176 & -544.881\\
\addlinespace
Kaski & 0.003 & 274.450 & -540.900 & -540.682 & -527.933\\
Rupandehi & 0.006 & 212.393 & -414.786 & -414.458 & -398.578\\
\bottomrule
\end{tabular}
\end{table}
\begin{table}

\caption{\label{tab:multiple-arima-summary}Model coefficients of ARIMA model for multiple log(price) series of major districtwise market hubs of Rice - Retail}
\centering
\begin{tabular}[t]{llrrrr}
\toprule
mktname & term & estimate & std.error & statistic & p.value\\
\midrule
Kathmandu & ma1 & 0.025 & 0.072 & 0.350 & 0.727\\
Kathmandu & ma2 & -0.353 & 0.074 & -4.789 & 0.000\\
Kathmandu & constant & 0.005 & 0.003 & 1.512 & 0.132\\
Parsa & ar1 & -0.198 & 0.104 & -1.912 & 0.057\\
Morang & ma1 & -0.403 & 0.079 & -5.096 & 0.000\\
\addlinespace
Morang & constant & 0.006 & 0.003 & 2.311 & 0.022\\
Kailali & ar1 & 1.387 & NaN & NaN & NaN\\
Kailali & ar2 & -0.485 & NaN & NaN & NaN\\
Kailali & ma1 & -1.579 & NaN & NaN & NaN\\
Kailali & ma2 & 0.617 & NaN & NaN & NaN\\
\addlinespace
Kailali & constant & 0.000 & 0.000 & 2.511 & 0.013\\
Banke & ar1 & 0.702 & 0.065 & 10.752 & 0.000\\
Banke & ma1 & -0.947 & 0.030 & -32.040 & 0.000\\
Banke & sar1 & -0.013 & 0.093 & -0.141 & 0.888\\
Banke & constant & 0.002 & 0.000 & 6.728 & 0.000\\
\addlinespace
Kaski & ma1 & -0.316 & 0.070 & -4.542 & 0.000\\
Kaski & constant & 0.005 & 0.002 & 2.233 & 0.027\\
Rupandehi & ma1 & -0.359 & 0.063 & -5.701 & 0.000\\
Rupandehi & sma1 & 0.100 & 0.064 & 1.557 & 0.121\\
\bottomrule
\end{tabular}
\end{table}
\begin{table}

\caption{\label{tab:multiple-arima-summary}Model coefficients of ARIMA model for multiple log(price) series of major districtwise market hubs of Wheat - Retail}
\centering
\begin{tabular}[t]{llrrrr}
\toprule
mktname & term & estimate & std.error & statistic & p.value\\
\midrule
Kathmandu & ma1 & -0.499 & 0.066 & -7.600 & 0.000\\
Kathmandu & constant & 0.005 & 0.002 & 1.901 & 0.059\\
Parsa & ma1 & -0.618 & 0.061 & -10.188 & 0.000\\
Parsa & constant & 0.006 & 0.002 & 2.906 & 0.004\\
Morang & ma1 & -0.235 & 0.072 & -3.249 & 0.001\\
\addlinespace
Morang & ma2 & -0.107 & 0.077 & -1.384 & 0.168\\
Morang & sar1 & 0.118 & 0.083 & 1.418 & 0.158\\
Morang & constant & 0.006 & 0.003 & 2.249 & 0.026\\
Kailali & ar1 & 0.276 & 0.148 & 1.865 & 0.063\\
Kailali & ma1 & -0.700 & 0.116 & -6.017 & 0.000\\
\addlinespace
Kailali & constant & 0.004 & 0.001 & 3.111 & 0.002\\
Banke & ma1 & -0.310 & 0.066 & -4.678 & 0.000\\
Banke & ma2 & -0.213 & 0.074 & -2.872 & 0.004\\
Banke & sar1 & -0.746 & 0.315 & -2.366 & 0.019\\
Banke & sar2 & -0.183 & 0.074 & -2.481 & 0.014\\
\addlinespace
Banke & sma1 & 0.647 & 0.323 & 2.002 & 0.047\\
Banke & constant & 0.012 & 0.003 & 3.299 & 0.001\\
Kaski & ar1 & -0.412 & 0.066 & -6.196 & 0.000\\
Kaski & sar1 & 0.093 & 0.097 & 0.954 & 0.341\\
Kaski & constant & 0.008 & 0.004 & 2.066 & 0.040\\
\addlinespace
Rupandehi & ar1 & -0.715 & 0.163 & -4.393 & 0.000\\
Rupandehi & ma1 & 0.403 & 0.158 & 2.554 & 0.011\\
Rupandehi & ma2 & -0.381 & 0.070 & -5.480 & 0.000\\
Rupandehi & sma1 & 0.215 & 0.075 & 2.854 & 0.005\\
\bottomrule
\end{tabular}
\end{table}

\hypertarget{var-model}{%
\section{VAR model}\label{var-model}}

\hypertarget{var}{%
\subsection{VAR}\label{var}}

VAR is a system regression model, i.e., there are more than one dependent variable. The regression is defined by a set of linear dynamic equations where each variable is specified as a function of an equal number of lags of itself and all other variables in the system. Any additional variable, adds to the modeling complexity by increasing an extra equation to be estimated.

The vector autoregression (VAR) model extends the idea of univariate autoregression to \(k\) time series regressions, where the lagged values of \emph{all} \(k\) series appear as regressors. Put differently, in a VAR model we regress a \emph{vector} of time series variables on lagged vectors of these variables. As for \(AR(p)\) models, the lag order is denoted by \(p\) so the \(VAR(p)\) model of two variables \(X_t\) and \(Y_t\) (\(k=2\)) is given by a vector of equations (Equation \ref{eqn:vector-regression-ts}).

\begin{equation}
\label{eqn:vector-regression-ts}
\begin{split}
\begin{aligned}
  Y_t =& \, \beta_{10} + \beta_{11} Y_{t-1} + \dots + \beta_{1p} Y_{t-p} + \gamma_{11} X_{t-1} + \dots + \gamma_{1p} X_{t-p} + u_{1t}, \\
  X_t =& \, \beta_{20} + \beta_{21} Y_{t-1} + \dots + \beta_{2p} Y_{t-p} + \gamma_{21} X_{t-1} + \dots + \gamma_{2p} X_{t-p} + u_{2t}.
\end{aligned}
\end{split}
\end{equation}

The \(\beta\)s and \(\gamma\)s can be estimated using OLS on each equation.

Simplifying this to a bivariate \(VAR(1)\), we can write the model in matrix form as:

\begin{equation}
\label{eqn:matix-var1-model}
Y_t = \beta_0 + \beta_1 Y_{t-1} + \mu_t
\end{equation}

Where,

\begin{itemize}
\tightlist
\item
  \(Y_t, Y_{t-1}\) and \(\mu_t\) are (2 x 1) column vectors
\item
  \(\beta_0\) is a (2 x 1) column vector
\item
  \(\beta_1\) is a (2 x 2) matrix
\end{itemize}

also,

\[
Y_t = 
\begin{pmatrix} 
y_{1t} \\
y_{2t}
\end{pmatrix},\ 
Y_{t-1} = 
\begin{pmatrix} 
y_{1t-1} \\
y_{2t-1}
\end{pmatrix}
\]

\[
\mu_t = 
\begin{pmatrix} 
\mu_{1t} \\
\mu_{2t}
\end{pmatrix},
\beta_{0} = 
\begin{pmatrix} 
\beta_{10} \\
\beta_{20}
\end{pmatrix},
\beta_{1} = 
\begin{pmatrix} 
\beta_{11} & \alpha_{11} \\
\alpha_{21} & \beta_{21}
\end{pmatrix}
\]

It is straightforward to estimate VAR models in \texttt{R}. A feasible approach is to simply use \texttt{lm()} for estimation of the individual equations. Furthermore, the \texttt{vars} package provides standard tools for estimation, diagnostic testing and prediction using this type of models.

Only when the assumptions presented below hold, the OLS estimators of the VAR coefficients are consistent and jointly normal in large samples so that the usual inferential methods such as confidence intervals and \(t\)-statistics can be used (Metcalfe and Cowpertwait \protect\hyperlink{ref-metcalfe2009introductory}{2009}).

Two series \(w_{x,t}\) and \(w_{y,t}\) are bivariate white noise if they are stationary and their cross-covariances \(\gamma_{xy}(k) = Cov(w_{x,t}, w_{y, t+k})\) satisfies

\[
\gamma_{xx}(k) = \gamma_{yy}(k) = \gamma_{xy}(k) = 0\ \text{for all } k \neq 0
\]

The parameters of a var(p) model can be estimated using the \texttt{ar} function in \texttt{R}, which selects a best-fitting order \(p\) based on the smallest information criterion values.

The structure of VARs also allows to jointly test restrictions across multiple equations. For instance, it may be of interest to test whether the coefficients on all regressors of the lag \(p\) are zero. This corresponds to testing the null that the lag order \(p-1\) is correct. Large sample joint normality of the coefficient estimates is convenient because it implies that we may simply use an \(F\)-test for this testing problem. The explicit formula for such a test statistic is rather complicated but fortunately such computations are easily done using the \texttt{ttcode("R")} functions we work with in this chapter. Just as in the case of a single equation, for a multiple equation model we choose the specification which has the smallest \(BIC(p)\), where

\[
\begin{aligned}
  BIC(p) =& \, \log\left[\text{det}(\widehat{\Sigma}_u)\right] + k(kp+1) \frac{\log(T)}{T}.
\end{aligned}
\]

with \(\widehat{\Sigma}_u\) denoting the estimate of the \(k \times k\) covariance matrix of the VAR errors and \(\text{det}(\cdot)\) denotes the determinant.

As for univariate distributed lag models, one should think carefully about variables to include in a VAR, as adding unrelated variables reduces the forecast accuracy by increasing the estimation error. This is particularly important because the number of parameters to be estimated grows qudratically to the number of variables modeled by the VAR.

\begin{longtable}[t]{lrrrrr}
\caption{\label{tab:retail-var-fit-tidy}Model performance indicators of VAR(AR(1)) model for selected districtwise market hub series in Rice and Wheat.}\\
\toprule
cmname-mktname & sigma2 & log\_lik & AIC & AICc & BIC\\
\midrule
Rice - Retail / Banke & 0.005987452 & 254.108 & -502.216 & -502.106 & -492.008\\
Rice - Retail / Kailali & 0.004562652 & 282.494 & -560.989 & -560.934 & -554.192\\
Rice - Retail / Kaski & 0.002558304 & 296.841 & -587.682 & -587.552 & -577.956\\
Rice - Retail / Kathmandu & 0.004799157 & 234.890 & -463.780 & -463.649 & -454.086\\
Rice - Retail / Morang & 0.004287195 & 248.052 & -490.103 & -489.973 & -480.378\\
\addlinespace
Rice - Retail / Parsa & 0.009750488 & 167.714 & -329.427 & -329.295 & -319.750\\
Rice - Retail / Rupandehi & 0.007998555 & 221.963 & -437.926 & -437.816 & -427.718\\
Wheat - Retail / Banke & 0.004799646 & 278.653 & -551.306 & -551.196 & -541.098\\
Wheat - Retail / Kailali & 0.004939438 & 275.466 & -544.932 & -544.822 & -534.724\\
Wheat - Retail / Kaski & 0.003879621 & 257.492 & -508.983 & -508.854 & -499.258\\
\addlinespace
Wheat - Retail / Kathmandu & 0.005504843 & 224.427 & -442.854 & -442.724 & -433.129\\
Wheat - Retail / Morang & 0.003155432 & 277.016 & -548.033 & -547.903 & -538.308\\
Wheat - Retail / Parsa & 0.007953767 & 186.655 & -367.310 & -367.178 & -357.633\\
Wheat - Retail / Rupandehi & 0.007253182 & 198.363 & -390.725 & -390.596 & -381.000\\
\bottomrule
\end{longtable}

\begin{longtable}[t]{llrrrr}
\caption{\label{tab:retail-var-fit-tidy}Model coefficients of VAR model for multiple log(price) series of major districtwise market hubs of Rice - Retail}\\
\toprule
mktname & term & estimate & std.error & statistic & p.value\\
\midrule
Banke & lag(lprice,1) & 0.982 & 0.012 & 82.933 & 0.000\\
Banke & constant & 0.062 & 0.038 & 1.626 & 0.105\\
Kailali & lag(lprice,1) & 1.001 & 0.001 & 723.506 & 0.000\\
Kaski & lag(lprice,1) & 0.985 & 0.010 & 99.189 & 0.000\\
Kaski & constant & 0.059 & 0.036 & 1.654 & 0.100\\
\addlinespace
Kathmandu & lag(lprice,1) & 0.971 & 0.015 & 63.567 & 0.000\\
Kathmandu & constant & 0.109 & 0.055 & 1.999 & 0.047\\
Morang & lag(lprice,1) & 0.973 & 0.014 & 71.100 & 0.000\\
Morang & constant & 0.098 & 0.046 & 2.121 & 0.035\\
Parsa & lag(lprice,1) & 0.963 & 0.021 & 46.349 & 0.000\\
\addlinespace
Parsa & constant & 0.125 & 0.070 & 1.774 & 0.078\\
Rupandehi & lag(lprice,1) & 0.972 & 0.016 & 62.573 & 0.000\\
Rupandehi & constant & 0.097 & 0.050 & 1.927 & 0.055\\
\bottomrule
\end{longtable}

\begin{longtable}[t]{llrrrr}
\caption{\label{tab:retail-var-fit-tidy}Model coefficients of VAR model for multiple log(price) series of major districtwise market hubs of Wheat - Retail}\\
\toprule
mktname & term & estimate & std.error & statistic & p.value\\
\midrule
Banke & lag(lprice,1) & 0.982 & 0.011 & 86.538 & 0.000\\
Banke & constant & 0.063 & 0.037 & 1.698 & 0.091\\
Kailali & lag(lprice,1) & 0.977 & 0.012 & 80.478 & 0.000\\
Kailali & constant & 0.080 & 0.039 & 2.033 & 0.043\\
Kaski & lag(lprice,1) & 0.974 & 0.013 & 72.988 & 0.000\\
\addlinespace
Kaski & constant & 0.097 & 0.047 & 2.046 & 0.042\\
Kathmandu & lag(lprice,1) & 0.972 & 0.017 & 57.967 & 0.000\\
Kathmandu & constant & 0.103 & 0.059 & 1.741 & 0.083\\
Morang & lag(lprice,1) & 0.983 & 0.012 & 79.565 & 0.000\\
Morang & constant & 0.063 & 0.043 & 1.492 & 0.137\\
\addlinespace
Parsa & lag(lprice,1) & 0.949 & 0.021 & 45.575 & 0.000\\
Parsa & constant & 0.178 & 0.071 & 2.514 & 0.013\\
Rupandehi & lag(lprice,1) & 0.955 & 0.021 & 46.210 & 0.000\\
Rupandehi & constant & 0.157 & 0.070 & 2.259 & 0.025\\
\bottomrule
\end{longtable}

\hypertarget{differenced-series}{%
\subsection{Differenced series}\label{differenced-series}}

\includegraphics{price_relations_of_food_cm_in_major_district_market_files/figure-latex/differenced-series-viz-1.pdf} \includegraphics{price_relations_of_food_cm_in_major_district_market_files/figure-latex/differenced-series-viz-2.pdf} \includegraphics{price_relations_of_food_cm_in_major_district_market_files/figure-latex/differenced-series-viz-3.pdf} \includegraphics{price_relations_of_food_cm_in_major_district_market_files/figure-latex/differenced-series-viz-4.pdf}

\hypertarget{causality-test}{%
\subsection{Causality test}\label{causality-test}}

Causality test is VAR based approach to explain cause-effect relationship among endogenous variables. However, the Granger-causality (Granger \protect\hyperlink{ref-granger1988causality}{1988}) inference does not, of course, establish the real causation phenomena. If one of the variables is sufficiently correlated to the other so that forecast of former depends on the later to a considerable extent, then the first variable is \emph{granger-caused} by the second one.

The Granger causality has been briefed to be useable test in certain cases of two series violating the stationarity assumption {[}\^{}{[}1{]}(\url{https://davegiles.blogspot.com/2011/04/testing-for-granger-causality.html}){]}. Papana et al. (\protect\hyperlink{ref-papana2014identifying}{2014}) state that GC test can only to applied if both the series are stationary{[}\footnote{Mindy Mallory's blog article also suggests that series be stationary}(\url{http://blog.mindymallory.com/2018/02/basic-time-series-analysis-the-var-model-explained/}){]}. Same paper also cautioned that VAR(1) models of cointegrated endogenous series will fail to capture long-run relationships. Therefore, the authors suggest surplus lag Granger-causality test be used if the series are a nonstationary data.

Below are the results of GC test showing consequences of using both stationarity and non stationary data, with bootstrapped confidence intervals.

\hypertarget{case-i-gc-test-on-undifferenced-logprice-series}{%
\subsubsection{Case I: GC test on undifferenced log(price) series}\label{case-i-gc-test-on-undifferenced-logprice-series}}

\begin{table}

\caption{\label{tab:gc-test-undifferenced-series}Granger causality test of multivariate (7 market series) var models of Rice log(price) series}
\centering
\begin{tabular}[t]{>{\raggedleft\arraybackslash}p{3em}>{\raggedleft\arraybackslash}p{3em}>{\raggedleft\arraybackslash}p{3em}>{\raggedleft\arraybackslash}p{20em}}
\toprule
statistic & p.value & parameter & method\\
\midrule
1.053 & 0.200 & 5000 & Granger causality H0: kathmandu do not Granger-cause parsa morang kailali banke kaski rupandehi\\
0.641 & 0.527 & 5000 & Granger causality H0: parsa do not Granger-cause kathmandu morang kailali banke kaski rupandehi\\
1.098 & 0.381 & 5000 & Granger causality H0: morang do not Granger-cause kathmandu parsa kailali banke kaski rupandehi\\
0.614 & 0.740 & 5000 & Granger causality H0: kailali do not Granger-cause kathmandu parsa morang banke kaski rupandehi\\
2.744 & 0.065 & 5000 & Granger causality H0: banke do not Granger-cause kathmandu parsa morang kailali kaski rupandehi\\
\addlinespace
2.637 & 0.062 & 5000 & Granger causality H0: kaski do not Granger-cause kathmandu parsa morang kailali banke rupandehi\\
1.262 & 0.171 & 5000 & Granger causality H0: rupandehi do not Granger-cause kathmandu parsa morang kailali banke kaski\\
\bottomrule
\end{tabular}
\end{table}

\begin{table}

\caption{\label{tab:gc-test-undifferenced-series}Granger causality test of multivariate (7 market series) var models of Wheat log(price) series}
\centering
\begin{tabular}[t]{>{\raggedleft\arraybackslash}p{3em}>{\raggedleft\arraybackslash}p{3em}>{\raggedleft\arraybackslash}p{3em}>{\raggedleft\arraybackslash}p{20em}}
\toprule
statistic & p.value & parameter & method\\
\midrule
2.555 & 0.029 & 5000 & Granger causality H0: kathmandu do not Granger-cause parsa morang kailali banke kaski rupandehi\\
0.646 & 0.738 & 5000 & Granger causality H0: parsa do not Granger-cause kathmandu morang kailali banke kaski rupandehi\\
2.865 & 0.037 & 5000 & Granger causality H0: morang do not Granger-cause kathmandu parsa kailali banke kaski rupandehi\\
1.684 & 0.204 & 5000 & Granger causality H0: kailali do not Granger-cause kathmandu parsa morang banke kaski rupandehi\\
1.086 & 0.452 & 5000 & Granger causality H0: banke do not Granger-cause kathmandu parsa morang kailali kaski rupandehi\\
\addlinespace
3.554 & 0.007 & 5000 & Granger causality H0: kaski do not Granger-cause kathmandu parsa morang kailali banke rupandehi\\
3.172 & 0.021 & 5000 & Granger causality H0: rupandehi do not Granger-cause kathmandu parsa morang kailali banke kaski\\
\bottomrule
\end{tabular}
\end{table}

\hypertarget{case-ii-gc-test-on-differenced-logprice-series}{%
\subsubsection{Case II: GC test on differenced log(price) series}\label{case-ii-gc-test-on-differenced-logprice-series}}

\begin{table}

\caption{\label{tab:gc-test-differenced-series}Granger causality test of multivariate (7 market series) var models of Rice first order differenced log(price) series}
\centering
\begin{tabular}[t]{>{\raggedleft\arraybackslash}p{3em}>{\raggedleft\arraybackslash}p{3em}>{\raggedleft\arraybackslash}p{3em}>{\raggedleft\arraybackslash}p{20em}}
\toprule
statistic & p.value & parameter & method\\
\midrule
0.388 & 0.641 & 5000 & Granger causality H0: kathmandu do not Granger-cause parsa morang kailali banke kaski rupandehi\\
1.610 & 0.147 & 5000 & Granger causality H0: parsa do not Granger-cause kathmandu morang kailali banke kaski rupandehi\\
2.020 & 0.218 & 5000 & Granger causality H0: morang do not Granger-cause kathmandu parsa kailali banke kaski rupandehi\\
1.713 & 0.085 & 5000 & Granger causality H0: kailali do not Granger-cause kathmandu parsa morang banke kaski rupandehi\\
0.622 & 0.511 & 5000 & Granger causality H0: banke do not Granger-cause kathmandu parsa morang kailali kaski rupandehi\\
\addlinespace
0.778 & 0.393 & 5000 & Granger causality H0: kaski do not Granger-cause kathmandu parsa morang kailali banke rupandehi\\
0.423 & 0.739 & 5000 & Granger causality H0: rupandehi do not Granger-cause kathmandu parsa morang kailali banke kaski\\
\bottomrule
\end{tabular}
\end{table}

\begin{table}

\caption{\label{tab:gc-test-differenced-series}Granger causality test of multivariate (7 market series) var models of Wheat first order differenced log(price) series}
\centering
\begin{tabular}[t]{>{\raggedleft\arraybackslash}p{3em}>{\raggedleft\arraybackslash}p{3em}>{\raggedleft\arraybackslash}p{3em}>{\raggedleft\arraybackslash}p{20em}}
\toprule
statistic & p.value & parameter & method\\
\midrule
0.787 & 0.504 & 5000 & Granger causality H0: kathmandu do not Granger-cause parsa morang kailali banke kaski rupandehi\\
1.165 & 0.235 & 5000 & Granger causality H0: parsa do not Granger-cause kathmandu morang kailali banke kaski rupandehi\\
1.788 & 0.097 & 5000 & Granger causality H0: morang do not Granger-cause kathmandu parsa kailali banke kaski rupandehi\\
0.961 & 0.422 & 5000 & Granger causality H0: kailali do not Granger-cause kathmandu parsa morang banke kaski rupandehi\\
1.942 & 0.139 & 5000 & Granger causality H0: banke do not Granger-cause kathmandu parsa morang kailali kaski rupandehi\\
\addlinespace
0.274 & 0.777 & 5000 & Granger causality H0: kaski do not Granger-cause kathmandu parsa morang kailali banke rupandehi\\
1.642 & 0.104 & 5000 & Granger causality H0: rupandehi do not Granger-cause kathmandu parsa morang kailali banke kaski\\
\bottomrule
\end{tabular}
\end{table}

\hypertarget{cointegration}{%
\section{Cointegration}\label{cointegration}}

\hypertarget{residual-based}{%
\subsection{Residual based}\label{residual-based}}

Since the food commodities are spatially linked, more of so because they occupy the same domestic market, it is obvious that factor affecting price of one inevitably affects other, especially that of same crop in a nearby market. Having evidence for nonstationarity, it is of interest to test for a common nonstationary component by means of a cointegration test (Non-stationarity is more valid for development regionwise price series).

A two step method proposed by Hylleberg et al. (\protect\hyperlink{ref-hylleberg1990seasonal}{1990}) can be used to test for cointegration.

The procedure simply regressess one series on the other and performs a unit root test on the residuals. This test is often named after Phillips, Ouliaris, and others (\protect\hyperlink{ref-phillips1990asymptotic}{1990}). Specifically, \texttt{po.test()} performs a Phillips-Perron test using an auxiliary regression without a constant and linear trend and the Newey-West estimator for the required long-run variance.

The test computes the Phillips-Ouliaris test for the null hypothesis that series is not cointegrated (Trapletti and Hornik \protect\hyperlink{ref-R-tseries}{2019}).

We check the rice retail price series for all combination major districtwise markets.

\begin{longtable}[t]{lrr}
\caption{\label{tab:pairwise-phillips-cointegration}Phillips-Ouliaris cointegration test for Rice log(price) series of selected district markethubs}\\
\toprule
combination & p\_value & statistic\\
\midrule
Morang-Parsa & 0.010 & -96.515\\
Morang-Kathmandu & 0.010 & -34.869\\
Morang-Kaski & 0.010 & -37.126\\
Morang-Rupandehi & 0.010 & -42.435\\
Morang-Banke & 0.010 & -65.794\\
\addlinespace
Morang-Kailali & 0.010 & -66.540\\
Parsa-Kathmandu & 0.010 & -59.925\\
Parsa-Kaski & 0.010 & -94.521\\
Parsa-Rupandehi & 0.010 & -63.256\\
Parsa-Banke & 0.010 & -91.480\\
\addlinespace
Parsa-Kailali & 0.010 & -89.453\\
Kathmandu-Kaski & 0.010 & -35.060\\
Kathmandu-Rupandehi & 0.010 & -37.100\\
Kathmandu-Banke & 0.010 & -34.602\\
Kathmandu-Kailali & 0.010 & -34.579\\
\addlinespace
Kaski-Rupandehi & 0.013 & -27.453\\
Kaski-Banke & 0.010 & -41.121\\
Kaski-Kailali & 0.010 & -52.289\\
Rupandehi-Banke & 0.010 & -89.619\\
Rupandehi-Kailali & 0.010 & -41.667\\
\addlinespace
Banke-Kailali & 0.010 & -58.835\\
\bottomrule
\end{longtable}

\begin{longtable}[t]{lrr}
\caption{\label{tab:pairwise-phillips-cointegration}Phillips-Ouliaris cointegration test for Wheat log(price) series of selected district markethubs}\\
\toprule
combination & p\_value & statistic\\
\midrule
Morang-Parsa & 0.01 & -79.626\\
Morang-Kathmandu & 0.01 & -53.276\\
Morang-Kaski & 0.01 & -71.222\\
Morang-Rupandehi & 0.01 & -79.788\\
Morang-Banke & 0.01 & -56.448\\
\addlinespace
Morang-Kailali & 0.01 & -55.875\\
Parsa-Kathmandu & 0.01 & -80.952\\
Parsa-Kaski & 0.01 & -90.071\\
Parsa-Rupandehi & 0.01 & -78.571\\
Parsa-Banke & 0.01 & -57.703\\
\addlinespace
Parsa-Kailali & 0.01 & -92.198\\
Kathmandu-Kaski & 0.01 & -82.232\\
Kathmandu-Rupandehi & 0.01 & -48.619\\
Kathmandu-Banke & 0.01 & -50.344\\
Kathmandu-Kailali & 0.01 & -73.156\\
\addlinespace
Kaski-Rupandehi & 0.01 & -58.758\\
Kaski-Banke & 0.01 & -49.955\\
Kaski-Kailali & 0.01 & -87.482\\
Rupandehi-Banke & 0.01 & -57.133\\
Rupandehi-Kailali & 0.01 & -64.151\\
\addlinespace
Banke-Kailali & 0.01 & -53.604\\
\bottomrule
\end{longtable}

Note \texttt{po.test} does not handle missing values, so we fix them through imputation. It is implemented through \texttt{tidyr::fill(...,\ .direction\ =\ "down")}.

The test suggests that all series (Both that of wheat and rice) are cointegrated for selected pairwise combination of district markets.

The problem with this approach is that it treats both series in an asymmetric fashion, while the concept of cointegration demands that the treatment be symmetric.

The po.test() function is testing the cointegration with Phillip's Z\_alpha test, which is the second residual-based test described by Phillips, Ouliaris, and others (\protect\hyperlink{ref-phillips1990asymptotic}{1990}). Because the po.test() will use the series at the first position to derive the residual used in the test, results would be determined by the series on the most left-hand side\footnote{\url{https://www.r-craft.org/r-news/phillips-ouliaris-test-for-cointegration/}}.

The Phillips-Ouliaris test implemented in the \texttt{ca.po()} function from the urca package is different. In the \texttt{ca.po()} function, there are two cointegration tests implemented, namely ``Pu'' and ``Pz'' tests. Although both the \texttt{ca.po()} function and the po.test() function are supposed to do the Phillips-Ouliaris test,outcomes from both functions are completely different.

Similar to Phillip's Z\_alpha test, the Pu test also is not invariant to the position of each series and therefore would give different outcomes based upon the series on the most left-hand side. On the contrary, the multivariate trace statistic of Pz test has its appeal in that the outcome won't change by the position of each series.

\hypertarget{var-based}{%
\subsection{VAR based}\label{var-based}}

The standard tests proceeding in a symmetric manner stem from Johansen's full-information maximum likelihood approach (Johansen \protect\hyperlink{ref-johansen1991estimation}{1991}).

A general vector autoregressive model is similar to the AR(p) model except that each quantity is vector valued and matrices are used as the coefficients. The general form of the VAR(p) model, without drift, is given by:

\begin{equation}
\label{eqn:var-general}
{\bf y_t} = {\bf \mu} + A_1 {\bf y_{t-1}} + \ldots + A_j {\bf y_{t-j}} + {\bf \varepsilon_t} 
\end{equation}

Where \({\bf \mu}\) is the vector-valued mean of the series, \(A_i\) are the coefficient matrices for each lag and \({\bf \varepsilon_t}\) is a multivariate Gaussian noise term with mean zero.

At this stage we can form a Vector Error Correction Model (VECM) by differencing the series (Equation \ref{eqn:vecm-differenced}).

\begin{equation}
\label{eqn:vecm-differenced}
\Delta {\bf y_t} = {\bf \mu} + A {\bf y_{t-1}} + \Gamma_1 \Delta {\bf y_{t-1}} + \ldots + \Gamma_j \Delta {\bf y_{t-j}} + {\bf \varepsilon_t} 
\end{equation}

Where \(\Delta {\bf y_t} = {\bf y_t} - {\bf y_{t-1}}\) is the differencing operator, \(A\) is the coefficient matrix for the first lag and \(\Gamma_i\) are the matrices for each differenced lag.

For a \(p^{th}\) -order cointegrated vector autoregressive (VAR) model, the error correction form is (omitting deterministic components; both no intercept or trend in either cointegrating equation or test var), we may rewrite the VAR in the form of Equation \ref{eqn:johansens} (Johansen \protect\hyperlink{ref-johansen1991estimation}{1991}).

\begin{equation}
\label{eqn:johansens}
\Delta y_t = \Pi y_{t-1} + \sum_{j = 1}^{p-1} {\Gamma_j \Delta y_{t-j}} + \varepsilon_t
\end{equation}

Where,

\[
\Pi = \sum^{p}_{i = 1}{A_{i}-I}; \Gamma = -\sum^{p}_{j = i + 1}{j}
\]

(Although, for simplicity sake, we assume absence of deterministic trends, there are five popular scenarios of including such trends in a cointegration test. All of these are described in (Johansen \protect\hyperlink{ref-johansen1995identifying}{1995}).)

Granger's representation theorem asserts that if the coefficient matrix \(\Pi\) has reduced rank \(r < k\), then there exist \(kxr\) matrices \(\alpha\) and \(\beta\) each with rank \(k\) such that \(\Pi = \alpha \beta^{\prime}\) and \(\beta^{\prime}y_t\) is \(I(0)\).

To achieve this an eigenvalue decomposition of \(A\) is carried out. The rank of the matrix \(A\) is given by \(r\) and the Johansen test sequentially tests whether this rank \(r\) is equal to zero, equal to one, through to \(r=n-1\), where \(n\) is the number of time series under test.

The null hypothesis of \(r=0\) means that there is no cointegration at all. A rank \(r > 0\) implies a cointegrating relationship between two or possibly more time series.

The eigenvalue decomposition results in a set of eigenvectors. The components of the largest eigenvector admits the important property of forming the coefficients of a linear combination of time series to produce a stationary portfolio. Notice how this differs from the CADF test (often known as the Engle-Granger procedure) where it is necessary to ascertain the linear combination a priori via linear regression and ordinary least squares (OLS).

In summary, the test checks for the situation of no cointegration, which occurs when the matrix \(A=0\). So, starting with the base value of \(r\) (i.e., \(r=0\)), if the test statistic is greater than critical values of at the 10\%, 5\% and 1\% levels, this would imply that we are \textbf{able} to reject the null of no cointegration. For the case r\textless{}=1, we if the calculated test statistic is below the critical values of, we are \textbf{unable} to reject the null, and the number of cointegrating vectors is between 0 and 1. The relevant tests are available in the function \texttt{urca::ca.jo()}. The basic version considers the eigenvalues of the matrix \(\Pi\) in the preceding equation.

Here, we employ the trace statistic -- the maximum eigenvalue, or ``lambdamax'' test is available as well -- in an equation amended by a constant term (specified by ecdet = ``const''), yielding:

Johansen cointegration test summary and time series plots for rice (district marketwise)

\begin{longtable}[t]{lrrrr}
\caption{\label{tab:rice-cajo-test}Johansen cointegration test summary for Morang-Parsa}\\
\toprule
gamma & test\_stat & 10pct & 5pct & 1pct\\
\midrule
r <= 1 | & 8.047588 & 7.52 & 9.24 & 12.97\\
r = 0  | & 29.352546 & 17.85 & 19.96 & 24.60\\
\bottomrule
\end{longtable}

\begin{longtable}[t]{lrrrr}
\caption{\label{tab:rice-cajo-test}Johansen cointegration test summary for Morang-Kathmandu}\\
\toprule
gamma & test\_stat & 10pct & 5pct & 1pct\\
\midrule
r <= 1 | & 7.080797 & 7.52 & 9.24 & 12.97\\
r = 0  | & 26.213687 & 17.85 & 19.96 & 24.60\\
\bottomrule
\end{longtable}

\begin{longtable}[t]{lrrrr}
\caption{\label{tab:rice-cajo-test}Johansen cointegration test summary for Morang-Kaski}\\
\toprule
gamma & test\_stat & 10pct & 5pct & 1pct\\
\midrule
r <= 1 | & 10.47841 & 7.52 & 9.24 & 12.97\\
r = 0  | & 26.58143 & 17.85 & 19.96 & 24.60\\
\bottomrule
\end{longtable}

\begin{longtable}[t]{lrrrr}
\caption{\label{tab:rice-cajo-test}Johansen cointegration test summary for Morang-Rupandehi}\\
\toprule
gamma & test\_stat & 10pct & 5pct & 1pct\\
\midrule
r <= 1 | & 9.529231 & 7.52 & 9.24 & 12.97\\
r = 0  | & 28.702528 & 17.85 & 19.96 & 24.60\\
\bottomrule
\end{longtable}

\begin{longtable}[t]{lrrrr}
\caption{\label{tab:rice-cajo-test}Johansen cointegration test summary for Morang-Banke}\\
\toprule
gamma & test\_stat & 10pct & 5pct & 1pct\\
\midrule
r <= 1 | & 11.07168 & 7.52 & 9.24 & 12.97\\
r = 0  | & 43.79799 & 17.85 & 19.96 & 24.60\\
\bottomrule
\end{longtable}

\begin{longtable}[t]{lrrrr}
\caption{\label{tab:rice-cajo-test}Johansen cointegration test summary for Morang-Kailali}\\
\toprule
gamma & test\_stat & 10pct & 5pct & 1pct\\
\midrule
r <= 1 | & 8.627746 & 7.52 & 9.24 & 12.97\\
r = 0  | & 37.159947 & 17.85 & 19.96 & 24.60\\
\bottomrule
\end{longtable}

\begin{longtable}[t]{lrrrr}
\caption{\label{tab:rice-cajo-test}Johansen cointegration test summary for Parsa-Kathmandu}\\
\toprule
gamma & test\_stat & 10pct & 5pct & 1pct\\
\midrule
r <= 1 | & 7.117339 & 7.52 & 9.24 & 12.97\\
r = 0  | & 28.121237 & 17.85 & 19.96 & 24.60\\
\bottomrule
\end{longtable}

\begin{longtable}[t]{lrrrr}
\caption{\label{tab:rice-cajo-test}Johansen cointegration test summary for Parsa-Kaski}\\
\toprule
gamma & test\_stat & 10pct & 5pct & 1pct\\
\midrule
r <= 1 | & 8.739686 & 7.52 & 9.24 & 12.97\\
r = 0  | & 33.345326 & 17.85 & 19.96 & 24.60\\
\bottomrule
\end{longtable}

\begin{longtable}[t]{lrrrr}
\caption{\label{tab:rice-cajo-test}Johansen cointegration test summary for Parsa-Rupandehi}\\
\toprule
gamma & test\_stat & 10pct & 5pct & 1pct\\
\midrule
r <= 1 | & 5.241234 & 7.52 & 9.24 & 12.97\\
r = 0  | & 21.237513 & 17.85 & 19.96 & 24.60\\
\bottomrule
\end{longtable}

\begin{longtable}[t]{lrrrr}
\caption{\label{tab:rice-cajo-test}Johansen cointegration test summary for Parsa-Banke}\\
\toprule
gamma & test\_stat & 10pct & 5pct & 1pct\\
\midrule
r <= 1 | & 8.905272 & 7.52 & 9.24 & 12.97\\
r = 0  | & 35.297439 & 17.85 & 19.96 & 24.60\\
\bottomrule
\end{longtable}

\begin{longtable}[t]{lrrrr}
\caption{\label{tab:rice-cajo-test}Johansen cointegration test summary for Parsa-Kailali}\\
\toprule
gamma & test\_stat & 10pct & 5pct & 1pct\\
\midrule
r <= 1 | & 6.477453 & 7.52 & 9.24 & 12.97\\
r = 0  | & 30.475019 & 17.85 & 19.96 & 24.60\\
\bottomrule
\end{longtable}

\begin{longtable}[t]{lrrrr}
\caption{\label{tab:rice-cajo-test}Johansen cointegration test summary for Kathmandu-Kaski}\\
\toprule
gamma & test\_stat & 10pct & 5pct & 1pct\\
\midrule
r <= 1 | & 6.896813 & 7.52 & 9.24 & 12.97\\
r = 0  | & 30.474234 & 17.85 & 19.96 & 24.60\\
\bottomrule
\end{longtable}

\begin{longtable}[t]{lrrrr}
\caption{\label{tab:rice-cajo-test}Johansen cointegration test summary for Kathmandu-Rupandehi}\\
\toprule
gamma & test\_stat & 10pct & 5pct & 1pct\\
\midrule
r <= 1 | & 5.706426 & 7.52 & 9.24 & 12.97\\
r = 0  | & 24.640127 & 17.85 & 19.96 & 24.60\\
\bottomrule
\end{longtable}

\begin{longtable}[t]{lrrrr}
\caption{\label{tab:rice-cajo-test}Johansen cointegration test summary for Kathmandu-Banke}\\
\toprule
gamma & test\_stat & 10pct & 5pct & 1pct\\
\midrule
r <= 1 | & 9.024811 & 7.52 & 9.24 & 12.97\\
r = 0  | & 30.343046 & 17.85 & 19.96 & 24.60\\
\bottomrule
\end{longtable}

\begin{longtable}[t]{lrrrr}
\caption{\label{tab:rice-cajo-test}Johansen cointegration test summary for Kathmandu-Kailali}\\
\toprule
gamma & test\_stat & 10pct & 5pct & 1pct\\
\midrule
r <= 1 | & 6.166887 & 7.52 & 9.24 & 12.97\\
r = 0  | & 27.654849 & 17.85 & 19.96 & 24.60\\
\bottomrule
\end{longtable}

\begin{longtable}[t]{lrrrr}
\caption{\label{tab:rice-cajo-test}Johansen cointegration test summary for Kaski-Rupandehi}\\
\toprule
gamma & test\_stat & 10pct & 5pct & 1pct\\
\midrule
r <= 1 | & 7.905289 & 7.52 & 9.24 & 12.97\\
r = 0  | & 22.551933 & 17.85 & 19.96 & 24.60\\
\bottomrule
\end{longtable}

\begin{longtable}[t]{lrrrr}
\caption{\label{tab:rice-cajo-test}Johansen cointegration test summary for Kaski-Banke}\\
\toprule
gamma & test\_stat & 10pct & 5pct & 1pct\\
\midrule
r <= 1 | & 8.927975 & 7.52 & 9.24 & 12.97\\
r = 0  | & 29.193633 & 17.85 & 19.96 & 24.60\\
\bottomrule
\end{longtable}

\begin{longtable}[t]{lrrrr}
\caption{\label{tab:rice-cajo-test}Johansen cointegration test summary for Kaski-Kailali}\\
\toprule
gamma & test\_stat & 10pct & 5pct & 1pct\\
\midrule
r <= 1 | & 7.744047 & 7.52 & 9.24 & 12.97\\
r = 0  | & 31.015706 & 17.85 & 19.96 & 24.60\\
\bottomrule
\end{longtable}

\begin{longtable}[t]{lrrrr}
\caption{\label{tab:rice-cajo-test}Johansen cointegration test summary for Rupandehi-Banke}\\
\toprule
gamma & test\_stat & 10pct & 5pct & 1pct\\
\midrule
r <= 1 | & 4.441382 & 7.52 & 9.24 & 12.97\\
r = 0  | & 46.391884 & 17.85 & 19.96 & 24.60\\
\bottomrule
\end{longtable}

\begin{longtable}[t]{lrrrr}
\caption{\label{tab:rice-cajo-test}Johansen cointegration test summary for Rupandehi-Kailali}\\
\toprule
gamma & test\_stat & 10pct & 5pct & 1pct\\
\midrule
r <= 1 | & 2.995481 & 7.52 & 9.24 & 12.97\\
r = 0  | & 20.488277 & 17.85 & 19.96 & 24.60\\
\bottomrule
\end{longtable}

\begin{longtable}[t]{lrrrr}
\caption{\label{tab:rice-cajo-test}Johansen cointegration test summary for Banke-Kailali}\\
\toprule
gamma & test\_stat & 10pct & 5pct & 1pct\\
\midrule
r <= 1 | & 2.788889 & 7.52 & 9.24 & 12.97\\
r = 0  | & 40.728755 & 17.85 & 19.96 & 24.60\\
\bottomrule
\end{longtable}

Johansen cointegration test summary and time series plots for wheat (district marketwise)

\begin{longtable}[t]{lrrrr}
\caption{\label{tab:wheat-cajo-test}Johansen cointegration test summary for Morang-Parsa}\\
\toprule
gamma & test\_stat & 10pct & 5pct & 1pct\\
\midrule
r <= 1 | & 8.67185 & 7.52 & 9.24 & 12.97\\
r = 0  | & 38.90466 & 17.85 & 19.96 & 24.60\\
\bottomrule
\end{longtable}

\begin{longtable}[t]{lrrrr}
\caption{\label{tab:wheat-cajo-test}Johansen cointegration test summary for Morang-Kathmandu}\\
\toprule
gamma & test\_stat & 10pct & 5pct & 1pct\\
\midrule
r <= 1 | & 6.03380 & 7.52 & 9.24 & 12.97\\
r = 0  | & 32.39395 & 17.85 & 19.96 & 24.60\\
\bottomrule
\end{longtable}

\begin{longtable}[t]{lrrrr}
\caption{\label{tab:wheat-cajo-test}Johansen cointegration test summary for Morang-Kaski}\\
\toprule
gamma & test\_stat & 10pct & 5pct & 1pct\\
\midrule
r <= 1 | & 14.80835 & 7.52 & 9.24 & 12.97\\
r = 0  | & 47.39019 & 17.85 & 19.96 & 24.60\\
\bottomrule
\end{longtable}

\begin{longtable}[t]{lrrrr}
\caption{\label{tab:wheat-cajo-test}Johansen cointegration test summary for Morang-Rupandehi}\\
\toprule
gamma & test\_stat & 10pct & 5pct & 1pct\\
\midrule
r <= 1 | & 6.698168 & 7.52 & 9.24 & 12.97\\
r = 0  | & 41.851350 & 17.85 & 19.96 & 24.60\\
\bottomrule
\end{longtable}

\begin{longtable}[t]{lrrrr}
\caption{\label{tab:wheat-cajo-test}Johansen cointegration test summary for Morang-Banke}\\
\toprule
gamma & test\_stat & 10pct & 5pct & 1pct\\
\midrule
r <= 1 | & 6.791701 & 7.52 & 9.24 & 12.97\\
r = 0  | & 29.981567 & 17.85 & 19.96 & 24.60\\
\bottomrule
\end{longtable}

\begin{longtable}[t]{lrrrr}
\caption{\label{tab:wheat-cajo-test}Johansen cointegration test summary for Morang-Kailali}\\
\toprule
gamma & test\_stat & 10pct & 5pct & 1pct\\
\midrule
r <= 1 | & 10.09031 & 7.52 & 9.24 & 12.97\\
r = 0  | & 34.84797 & 17.85 & 19.96 & 24.60\\
\bottomrule
\end{longtable}

\begin{longtable}[t]{lrrrr}
\caption{\label{tab:wheat-cajo-test}Johansen cointegration test summary for Parsa-Kathmandu}\\
\toprule
gamma & test\_stat & 10pct & 5pct & 1pct\\
\midrule
r <= 1 | & 5.572257 & 7.52 & 9.24 & 12.97\\
r = 0  | & 39.875312 & 17.85 & 19.96 & 24.60\\
\bottomrule
\end{longtable}

\begin{longtable}[t]{lrrrr}
\caption{\label{tab:wheat-cajo-test}Johansen cointegration test summary for Parsa-Kaski}\\
\toprule
gamma & test\_stat & 10pct & 5pct & 1pct\\
\midrule
r <= 1 | & 9.047446 & 7.52 & 9.24 & 12.97\\
r = 0  | & 43.549450 & 17.85 & 19.96 & 24.60\\
\bottomrule
\end{longtable}

\begin{longtable}[t]{lrrrr}
\caption{\label{tab:wheat-cajo-test}Johansen cointegration test summary for Parsa-Rupandehi}\\
\toprule
gamma & test\_stat & 10pct & 5pct & 1pct\\
\midrule
r <= 1 | & 7.098776 & 7.52 & 9.24 & 12.97\\
r = 0  | & 35.167352 & 17.85 & 19.96 & 24.60\\
\bottomrule
\end{longtable}

\begin{longtable}[t]{lrrrr}
\caption{\label{tab:wheat-cajo-test}Johansen cointegration test summary for Parsa-Banke}\\
\toprule
gamma & test\_stat & 10pct & 5pct & 1pct\\
\midrule
r <= 1 | & 7.52789 & 7.52 & 9.24 & 12.97\\
r = 0  | & 36.13913 & 17.85 & 19.96 & 24.60\\
\bottomrule
\end{longtable}

\begin{longtable}[t]{lrrrr}
\caption{\label{tab:wheat-cajo-test}Johansen cointegration test summary for Parsa-Kailali}\\
\toprule
gamma & test\_stat & 10pct & 5pct & 1pct\\
\midrule
r <= 1 | & 8.944134 & 7.52 & 9.24 & 12.97\\
r = 0  | & 48.858929 & 17.85 & 19.96 & 24.60\\
\bottomrule
\end{longtable}

\begin{longtable}[t]{lrrrr}
\caption{\label{tab:wheat-cajo-test}Johansen cointegration test summary for Kathmandu-Kaski}\\
\toprule
gamma & test\_stat & 10pct & 5pct & 1pct\\
\midrule
r <= 1 | & 9.856107 & 7.52 & 9.24 & 12.97\\
r = 0  | & 48.980703 & 17.85 & 19.96 & 24.60\\
\bottomrule
\end{longtable}

\begin{longtable}[t]{lrrrr}
\caption{\label{tab:wheat-cajo-test}Johansen cointegration test summary for Kathmandu-Rupandehi}\\
\toprule
gamma & test\_stat & 10pct & 5pct & 1pct\\
\midrule
r <= 1 | & 4.913668 & 7.52 & 9.24 & 12.97\\
r = 0  | & 27.111863 & 17.85 & 19.96 & 24.60\\
\bottomrule
\end{longtable}

\begin{longtable}[t]{lrrrr}
\caption{\label{tab:wheat-cajo-test}Johansen cointegration test summary for Kathmandu-Banke}\\
\toprule
gamma & test\_stat & 10pct & 5pct & 1pct\\
\midrule
r <= 1 | & 5.079637 & 7.52 & 9.24 & 12.97\\
r = 0  | & 27.171088 & 17.85 & 19.96 & 24.60\\
\bottomrule
\end{longtable}

\begin{longtable}[t]{lrrrr}
\caption{\label{tab:wheat-cajo-test}Johansen cointegration test summary for Kathmandu-Kailali}\\
\toprule
gamma & test\_stat & 10pct & 5pct & 1pct\\
\midrule
r <= 1 | & 6.417786 & 7.52 & 9.24 & 12.97\\
r = 0  | & 35.001568 & 17.85 & 19.96 & 24.60\\
\bottomrule
\end{longtable}

\begin{longtable}[t]{lrrrr}
\caption{\label{tab:wheat-cajo-test}Johansen cointegration test summary for Kaski-Rupandehi}\\
\toprule
gamma & test\_stat & 10pct & 5pct & 1pct\\
\midrule
r <= 1 | & 10.27943 & 7.52 & 9.24 & 12.97\\
r = 0  | & 33.21418 & 17.85 & 19.96 & 24.60\\
\bottomrule
\end{longtable}

\begin{longtable}[t]{lrrrr}
\caption{\label{tab:wheat-cajo-test}Johansen cointegration test summary for Kaski-Banke}\\
\toprule
gamma & test\_stat & 10pct & 5pct & 1pct\\
\midrule
r <= 1 | & 8.023414 & 7.52 & 9.24 & 12.97\\
r = 0  | & 32.713727 & 17.85 & 19.96 & 24.60\\
\bottomrule
\end{longtable}

\begin{longtable}[t]{lrrrr}
\caption{\label{tab:wheat-cajo-test}Johansen cointegration test summary for Kaski-Kailali}\\
\toprule
gamma & test\_stat & 10pct & 5pct & 1pct\\
\midrule
r <= 1 | & 11.36236 & 7.52 & 9.24 & 12.97\\
r = 0  | & 40.97431 & 17.85 & 19.96 & 24.60\\
\bottomrule
\end{longtable}

\begin{longtable}[t]{lrrrr}
\caption{\label{tab:wheat-cajo-test}Johansen cointegration test summary for Rupandehi-Banke}\\
\toprule
gamma & test\_stat & 10pct & 5pct & 1pct\\
\midrule
r <= 1 | & 6.253642 & 7.52 & 9.24 & 12.97\\
r = 0  | & 34.339634 & 17.85 & 19.96 & 24.60\\
\bottomrule
\end{longtable}

\begin{longtable}[t]{lrrrr}
\caption{\label{tab:wheat-cajo-test}Johansen cointegration test summary for Rupandehi-Kailali}\\
\toprule
gamma & test\_stat & 10pct & 5pct & 1pct\\
\midrule
r <= 1 | & 7.004144 & 7.52 & 9.24 & 12.97\\
r = 0  | & 31.782559 & 17.85 & 19.96 & 24.60\\
\bottomrule
\end{longtable}

\begin{longtable}[t]{lrrrr}
\caption{\label{tab:wheat-cajo-test}Johansen cointegration test summary for Banke-Kailali}\\
\toprule
gamma & test\_stat & 10pct & 5pct & 1pct\\
\midrule
r <= 1 | & 7.875127 & 7.52 & 9.24 & 12.97\\
r = 0  | & 32.936232 & 17.85 & 19.96 & 24.60\\
\bottomrule
\end{longtable}

Wheat series and cointegration plots

\begin{center}\includegraphics[width=1\linewidth]{price_relations_of_food_cm_in_major_district_market_files/figure-latex/wheat-coint-plots-1} \includegraphics[width=1\linewidth]{price_relations_of_food_cm_in_major_district_market_files/figure-latex/wheat-coint-plots-2} \end{center}

Rice series and cointegration plots

\begin{center}\includegraphics[width=1\linewidth]{price_relations_of_food_cm_in_major_district_market_files/figure-latex/rice-coint-plots-1} \includegraphics[width=1\linewidth]{price_relations_of_food_cm_in_major_district_market_files/figure-latex/rice-coint-plots-2} \end{center}

\hypertarget{bibliography}{%
\section*{Bibliography}\label{bibliography}}
\addcontentsline{toc}{section}{Bibliography}

\hypertarget{refs}{}
\leavevmode\hypertarget{ref-granger1988causality}{}%
Granger, Clive WJ. 1988. ``Causality, Cointegration, and Control.'' \emph{Journal of Economic Dynamics and Control} 12 (2-3): 551--59.

\leavevmode\hypertarget{ref-hylleberg1990seasonal}{}%
Hylleberg, Svend, Robert F Engle, Clive WJ Granger, and Byung Sam Yoo. 1990. ``Seasonal Integration and Cointegration.'' \emph{Journal of Econometrics} 44 (1-2): 215--38.

\leavevmode\hypertarget{ref-johansen1991estimation}{}%
Johansen, Søren. 1991. ``Estimation and Hypothesis Testing of Cointegration Vectors in Gaussian Vector Autoregressive Models.'' \emph{Econometrica: Journal of the Econometric Society}, 1551--80.

\leavevmode\hypertarget{ref-johansen1995identifying}{}%
---------. 1995. ``Identifying Restrictions of Linear Equations with Applications to Simultaneous Equations and Cointegration.'' \emph{Journal of Econometrics} 69 (1): 111--32.

\leavevmode\hypertarget{ref-metcalfe2009introductory}{}%
Metcalfe, Andrew V, and Paul SP Cowpertwait. 2009. \emph{Introductory Time Series with R}. Springer.

\leavevmode\hypertarget{ref-papana2014identifying}{}%
Papana, Angeliki, Catherine Kyrtsou, Dimitris Kugiumtzis, Cees Diks, and others. 2014. ``Identifying Causal Relationships in Case of Non-Stationary Time Series.'' \emph{Thessaloniki: Department of Economics of the University of Macedonia}.

\leavevmode\hypertarget{ref-phillips1990asymptotic}{}%
Phillips, Peter CB, Sam Ouliaris, and others. 1990. ``Asymptotic Properties of Residual Based Tests for Cointegration.'' \emph{Econometrica} 58 (1): 165--93.

\leavevmode\hypertarget{ref-R-tseries}{}%
Trapletti, Adrian, and Kurt Hornik. 2019. \emph{Tseries: Time Series Analysis and Computational Finance}. \url{https://CRAN.R-project.org/package=tseries}.

\leavevmode\hypertarget{ref-woodward2017applied}{}%
Woodward, Wayne A, Henry L Gray, and Alan C Elliott. 2017. \emph{Applied Time Series Analysis with R}. CRC press.

\end{document}
